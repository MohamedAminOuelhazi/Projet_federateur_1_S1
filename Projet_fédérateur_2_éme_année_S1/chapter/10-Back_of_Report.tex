\newpage
\thispagestyle{empty}

%Français
\fontsize{12pt}{12pt}\selectfont
\begin{spacing}{1.15}

\begin{itemize}
    \item \underline{\textbf{{\huge R}ésumé:}}\\
    Ce projet réalisé au sein de l’entreprise BeeCoders vise à mettre en place une solution automatique de gestion d’un restaurant via QR-code. La plateforme en ligne a pour objectif de simplifier et d'optimiser la gestion des commandes dans un restaurant. Scrum a été choisi comme méthodologie de développement, tandis qu’Angular a été utilisé pour le frontend et Spring Boot pour le backend. L’environnement de développement préféré pour ce projet était Visual Studio Code (VS Code). La phase de conception a impliqué l’utilisation d’UML pour modéliser l’architecture du système, décrire les interactions entre les composantes, et créer des diagrammes de classes, de séquence et de cas d’utilisation. Draw a été l’outil privilégié pour créer ces diagrammes de manière efficace et professionnelle. En ce qui concerne les langages de programmation, TypeScript a été principalement utilisé ainsi que JavaScript. Pour le design et la mise en page, Bootstrap, un framework CSS populaire, a été utilisé pour créer des interfaces utilisateur. HTML a été employé pour structurer les pages et CSS pour le style. La gestion des données a été réalisée en utilisant le système de gestion de base de données relationnelle MySQL. Pour faciliter la création et la gestion de la base de données tout au long du projet, un environnement de développement local a été mis en place avec XAMPP, incluant un serveur Apache. \par
\end{itemize}

\underline{\textbf{Mots clés:}} QR-code, scrum, angular, spring boot, MySQL, XAMPP.\par
\rule{425pt}{0.75pt}

\begin{itemize}
    \item \underline{\textbf{{\huge A}bstract :}}\\
    This project carried out within the company BeeCoders aims to implement an automated restaurant management solution via QR-code. The online platform aims to simplify and optimize order management in a restaurant. Scrum was chosen as the development methodology, while Angular was used for the frontend and Spring Boot for the backend. The preferred development environment for this project was Visual Studio Code (VS Code). The design phase involved the use of UML to model the system architecture, describe the interactions between components, and create class, sequence, and use case diagrams. Draw was the preferred tool for creating these diagrams effectively and professionally. Regarding programming languages, TypeScript was mainly used along with JavaScript. For design and layout, Bootstrap, a popular CSS framework, was used to create user interfaces. HTML was employed to structure the pages and CSS for styling. Data management was carried out using the MySQL relational database management system. To facilitate the creation and management of the database throughout the project, a local development environment was set up with XAMPP, including an Apache server. \par
\end{itemize}

\underline{\textbf{Key-words:}} QR-code, scrum, angular, spring boot, MySQL, XAMPP.

\end{spacing}
