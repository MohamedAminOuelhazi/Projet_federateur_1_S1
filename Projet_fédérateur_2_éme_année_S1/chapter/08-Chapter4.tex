\chapter{CHAPITRE 4 : \textbf{Workflow de pv } }
\begin{spacing}{1.2}
\minitoc
\thispagestyle{MyStyle}
\end{spacing}
\newpage

\section{INTRODUCTION}
Dans ce chapitre, nous présentons le deuxiéme sprint intitulé "Gestion des espaces client et serveur", nous allons commencer par la présentation de backlog du sprint 2, suivie la spécification fonctionnelle et la conception. Enfin, nous illustrerons ce sprint à l'aide de captures d'écran pour une meilleure compréhension.

\section{Backlog de sprint 2 }
Voici  le tableau \ref{tableau:Backlog du sprint 2} du backlog du sprint 2 : 

\begin{longtable}{|p{3.5cm}|p{9cm}|p{1.5cm}|p{2cm}|}
\caption{Backlog Produit} %

\label{tableau:Backlog Produit}
\\
\hline
 \cellcolor[HTML]{FFF2CC}Fonctionnalités  
& \cellcolor[HTML]{FFF2CC}User Stories   
& \cellcolor[HTML]{FFF2CC}Priorités
& \cellcolor[HTML]{FFF2CC}Estimation\\
\hline
\multirow{4}{8em}{Gérer pv}
& En tant que FTE, je veux créer un pv & Élevée& 4 jours
\\\cline{2-4}
& En tant que FTE, je veux consulter un pv & Moyenne& 3 jours
\\\cline{2-4}
& En tant que DT, je veux consulter un pv & Moyenne& 1 jours
\\\cline{2-4}
& En tant que DT, je veux signer un pv & Moyenne& 1 jours
\\\cline{2-4}
& En tant que DT, je veux rejeter un pv & Moyenne& 1 jours
\\\cline{2-4}
& En tant que CT, je veux consulter un pv & Moyenne& 1 jours
\\\cline{2-4}
& En tant que CT, je veux signer un pv & Moyenne& 1 jours
\\\cline{2-4}
& En tant que CT, je veux rejeter un pv & Moyenne& 1 jours
\\\cline{2-4}
& En tant que FTE, je veux uploader une nouvelle version de pv & Moyenne& 1 jours
\\\hline
\end{longtable}

Chaque user story représente une fonctionnalité ou une action que le FTE, DT ou le CT  souhaite pouvoir réaliser dans le système. Les priorités sont classées en élevée, moyenne ou faible, reflétant l'importance relative de chaque user story. 

\section{Spécifications fonctionnelles }
Dans cette partie, nous présentons le digramme de cas d'uilisation du sprint 2  ainsi que les description textuelles. 

\subsection{Diagramme de cas d'utilisation du sprint 2 }
Le diagramme de cas d'utilisation du sprint 2, présenté dans la figure 4.1, illustre les besoins fonctionnels sous la forme d'interactions entre le système et les acteurs (serveur et client).
\begin{figure}[H]%
    \center%
    \setlength{\fboxsep}{5pt}%
    \setlength{\fboxrule}{0.5pt}%
    \includegraphics[width=15cm,height=14cm]{images/class digram sprint 2.drawio.png}%
        \caption{Diagramme de cas d'utilisation du sprint 2 }%

\end{figure}

\subsection{Description textuelles}

 L’objectif de cette activité est de décrire textuellement les scénarios des cas d’utilisation. Il faut préciser comment chaque scénario commence, comment il se termine et comment l’acteur interagit avec l’application web.  

\textbf{Description textuelle du cas d'utilisation "Créer un PV" :}\\
Le tableau \ref{tableau:Description textuelle du cas d'utilisation "Créer un PV"} présente la description textuelle du cas d’utilisation "Créer un PV", où le FTE crée un procès-verbal et le soumet à signature. 
\begin{longtable}{|p{5cm}|p{10cm}|} 
\caption{Description textuelle du cas d'utilisation "Créer un PV"} % 
\label{tableau:Description textuelle du cas d'utilisation "Créer un PV"}
\\
\hline
 \cellcolor[HTML]{FFF2CC}Cas d'utiliation  
& \cellcolor[HTML]{FFF2CC}Créer un PV\\
\hline
 Acteur & FTE 
\\ \hline
 Précondition  & 
\begin{enumerate}
    \item L'utilisateur est authentifié avec le rôle FTE.
    \item Une réunion associée existe (statut VALIDATED).
\end{enumerate}
\\\hline
Post-condition & 
\begin{enumerate}
    \item Un nouveau PV (version \texttt{v1}) est créé avec le statut \textit{PENDING} (en attente de signature).
    \item Une notification est envoyée à la Direction Technique (DT) et aux membres de la Commission Technique (CT) concernés.
\end{enumerate}
\\ \hline
Scénario principal & 
\begin{enumerate}
    \item \textbf{Accès :} Le FTE ouvre le module \textit{PV}.
    \item \textbf{Création :} Le FTE clique sur \textit{Nouveau PV}, sélectionne la réunion, puis renseigne les champs.
    \item \textbf{Soumission :} Le FTE soumet le PV pour signature. Le statut passe à \textit{PENDING} et des notifications sont émises.
\end{enumerate}
\\ \hline
Scénario alternatif& 
\begin{enumerate}
    \item \textbf Validation formulaire échoue \(\rightarrow\) message d'erreur, corrections requises.
    \item \textbf Pièce jointe trop volumineuse \(\rightarrow\) échec d'upload, suggestion de compresser le fichier.
\end{enumerate}
\\\hline
\end{longtable}

\textbf{Description textuelle du cas d'utilisation "Consulter un PV" :}\\
Le tableau \ref{tableau:Description textuelle du cas d'utilisation "Consulter un PV"} présente la description textuelle du cas d’utilisation "Consulter un PV", où un utilisateur (FTE, DT ou CT) accède au contenu d’un PV. 
\begin{longtable}{|p{5cm}|p{10cm}|} 
\caption{Description textuelle du cas d'utilisation "Consulter un PV"} % 
\label{tableau:Description textuelle du cas d'utilisation "Consulter un PV"}
\\
\hline
 \cellcolor[HTML]{FFF2CC}Cas d'utiliation  
& \cellcolor[HTML]{FFF2CC}Consulter un PV\\
\hline
 Acteur & FTE, DT, CT 
\\ \hline
 Précondition  & 
\begin{enumerate}
    \item L'utilisateur est authentifié.
    \item L'utilisateur possède les droits d'accès au PV (membre de la réunion).
\end{enumerate}
\\\hline
Post-condition & 
\begin{enumerate}
    \item Le PV est affiché dans le visualiseur (avec historique des versions).
    \item L'accès est journalisé (audit).
    \item Si le PV est \textit{ACCEPTED} par tous, le téléchargement est possible pour tous.
\end{enumerate}
\\ \hline
Scénario principal & 
\begin{enumerate}
    \item \textbf{Accès :} L'utilisateur ouvre le module \textit{Réunion}.
    \item \textbf{Choisir :} Il cliquer sur réunion correspondante de pv que je vais voir.
    \item \textbf{Affichage :} Il sélectionne un PV et consulte son contenu, les pièces jointes et l'état des signatures.
    \item \textbf{Téléchargement (optionnel) :} Si toutes les signatures sont collectées, il peut télécharger le PDF signé.
\end{enumerate}
\\ \hline
Scénario alternatif& 
\begin{enumerate}
    \item  Accès refusé (droits insuffisants) \(\rightarrow\) message d'erreur.
    \item PV introuvable  \(\rightarrow\) message d'information.
\end{enumerate}
\\\hline
\end{longtable}


\textbf{Description textuelle du cas d'utilisation "Signer un PV" :}\\
Le tableau \ref{tableau:Description textuelle du cas d'utilisation "Signer un PV"} présente la description textuelle du cas d’utilisation "Signer un PV", où un membre DT/CT signe électroniquement un PV via signature manuscrite (ngx-signature-pad). 
\begin{longtable}{|p{5cm}|p{10cm}|} 
\caption{Description textuelle du cas d'utilisation "Signer un PV"} % 
\label{tableau:Description textuelle du cas d'utilisation "Signer un PV"}
\\
\hline
 \cellcolor[HTML]{FFF2CC}Cas d'utiliation  
& \cellcolor[HTML]{FFF2CC}Signer un PV\\
\hline
 Acteur & DT ou CT 
\\ \hline
 Précondition  & 
\begin{enumerate}
    \item L'utilisateur est authentifié avec le rôle DT ou CT.
    \item Le PV ciblé est au statut \textit{PENDING}.
    \item L'utilisateur fait partie de la liste des signataires.
\end{enumerate}
\\\hline
Post-condition & 
\begin{enumerate}
    \item La signature est enregistrée et horodatée ; l'état du signataire passe à \textit{ACCEPTED}.
    \item Si tous les signataires ont accepté, le PV devient téléchargeable par tous (PDF consolidé).
    \item Des notifications sont envoyées (au FTE).
\end{enumerate}
\\ \hline
Scénario principal & 
\begin{enumerate}
    \item \textbf{Ouverture :} Le signataire ouvre le PV depuis la liste.
    \item \textbf{Revue :} Il consulte le contenu et les pièces jointes.
    \item \textbf{Signature :} Il clique sur \textit{Signer}, dessine sa signature, confirme.
    \item \textbf{Confirmation :} Le système enregistre la signature et met à jour le tableau des signataires.
\end{enumerate}
\\ \hline
Scénario alternatif& 
\begin{enumerate}
    \item Le signataire choisit \textit{Refuser} (voir cas d'utilisation dédié).
    \item Signature invalide ou zone vide \(\rightarrow\) message d'erreur, nouvelle tentative.
\end{enumerate}
\\\hline
\end{longtable}


\textbf{Description textuelle du cas d'utilisation "Rejeter un PV" :}\\
Le tableau \ref{tableau:Description textuelle du cas d'utilisation "Rejeter un PV"} présente la description textuelle du cas d’utilisation "Rejeter un PV", où un membre DT/CT refuse la validation d’un PV avec un motif. 
\begin{longtable}{|p{5cm}|p{10cm}|} 
\caption{Description textuelle du cas d'utilisation "Rejeter un PV"} % 
\label{tableau:Description textuelle du cas d'utilisation "Rejeter un PV"}
\\
\hline
 \cellcolor[HTML]{FFF2CC}Cas d'utiliation  
& \cellcolor[HTML]{FFF2CC}Rejeter un PV\\
\hline
 Acteur & DT ou CT 
\\ \hline
 Précondition  & 
\begin{enumerate}
    \item L'utilisateur est authentifié (rôle DT/CT).
    \item Le PV est au statut \textit{PENDING}.
    \item L'utilisateur est un signataire attendu.
\end{enumerate}
\\\hline
Post-condition & 
\begin{enumerate}
    \item L'état du signataire passe à \textit{REJECTED} avec un commentaire obligatoire.
    \item Le PV global passe à \textit{REJECTED}
    \item Une notification détaillant le motif est envoyée au FTE.
\end{enumerate}
\\ \hline
Scénario principal & 
\begin{enumerate}
    \item \textbf{Ouverture :} Le signataire ouvre le PV.
    \item \textbf{Décision :} Il clique \textit{Refuser}, saisit le motif dans le champ commentaire.
    \item \textbf{Soumission :} Il confirme le refus.
    \item \textbf{Notifications :} Le FTE reçoit une notification avec le motif de rejet.
\end{enumerate}
\\ \hline
Scénario alternatif& 
\begin{enumerate}
    \item Motif vide \(\rightarrow\) validation refuse la soumission.
\end{enumerate}
\\\hline
\end{longtable}

\textbf{Description textuelle du cas d'utilisation "Uploader une nouvelle version de PV" :}\\
Le tableau \ref{tableau:Description textuelle du cas d'utilisation "Uploader une nouvelle version de PV"} présente la description textuelle du cas d’utilisation "Uploader une nouvelle version de PV", où le FTE dépose une version corrigée suite à un rejet. 
\begin{longtable}{|p{5cm}|p{10cm}|} 
\caption{Description textuelle du cas d'utilisation "Uploader une nouvelle version de PV"} % 
\label{tableau:Description textuelle du cas d'utilisation "Uploader une nouvelle version de PV"}
\\
\hline
 \cellcolor[HTML]{FFF2CC}Cas d'utiliation  
& \cellcolor[HTML]{FFF2CC}Uploader une nouvelle version de PV\\
\hline
 Acteur & FTE 
\\ \hline
 Précondition  & 
\begin{enumerate}
    \item Le PV courant a été \textit{REJECTED} (ou une révision a été demandée).
    \item Le FTE est authentifié et autorisé sur le PV.
\end{enumerate}
\\\hline
Post-condition & 
\begin{enumerate}
    \item Une nouvelle version \texttt{v(n+1)} remplace la précédente comme version active.
    \item Le statut du PV repasse à \textit{PENDING} et réinitialise les états de signature.
    \item Des notifications sont envoyées aux signataires (DT/CT) pour revue et signature de la nouvelle version.
\end{enumerate}
\\ \hline
Scénario principal & 
\begin{enumerate}
    \item \textbf{Ouverture :} Le FTE ouvre le PV rejeté et clique \textit{Nouvelle version}.
    \item \textbf{Mise à jour :} Il modifie le contenu ou uploade un fichier révisé.
    \item \textbf{Soumission :} Il soumet la version \texttt{v(n+1)} pour signature (\textit{PENDING}).
    \item \textbf{Notifications :} Le système notifie DT/CT et met à jour l'historique des versions.
\end{enumerate}
\\ \hline
Scénario alternatif& 
\begin{enumerate}
    \item Fichier non conforme (taille/format) \(\rightarrow\) upload refusé, instructions fournies.
\end{enumerate}
\\\hline
\end{longtable}


\vspace{100pt}

\section{Conception }
Dans le processus de conception, il est crucial d'analyser les différentes interactions et fonctionnalités du système en utilisant des outils et des techniques spécifiques. Nous commencerons par présenter les diagrammes de séquences des cas d'utilisation qui ont été précédemment affinés. Ensuite, nous montrerons le diagramme de classes de  la deuxième sprint.\par
\subsection{Diagrammes de séquences }

\begin{itemize}
    \item \textbf{Diagramme de séquences "Créer un PV" :}

Le diagramme de séquences "Créer un PV", présenté dans la figure X.X, décrit le processus par lequel le FTE peut créer un procès-verbal et le soumettre pour signature. 

\begin{figure}[H]%
    \center%
    \setlength{\fboxsep}{5pt}%
    \setlength{\fboxrule}{0.5pt}%
    \includegraphics[width=17cm,height=18cm]{images/créer un pv.png}%
    \caption{Diagramme de séquences "Créer un PV"}%
\end{figure}

\item \textbf{Diagramme de séquences "Consulter un PV" :}

Le diagramme de séquences "Consulter un PV", présenté dans la figure X.X, décrit le processus par lequel un utilisateur (FTE, DT ou CT) peut consulter le contenu d’un procès-verbal.  

\begin{figure}[H]%
    \center%
    \setlength{\fboxsep}{5pt}%
    \setlength{\fboxrule}{0.5pt}%
    \includegraphics[width=17cm,height=18cm]{images/consulter un pv.png}%
    \caption{Diagramme de séquences "Consulter un PV"}%
\end{figure}

\item \textbf{Diagramme de séquences "Signer un PV" :}

Le diagramme de séquences "Signer un PV", présenté dans la figure X.X, détaille la manière dont un membre de la Direction Technique (DT) ou de la Commission Technique (CT) peut signer électroniquement un PV.  

\begin{figure}[H]%
    \center%
    \setlength{\fboxsep}{5pt}%
    \setlength{\fboxrule}{0.5pt}%
    \includegraphics[width=17cm,height=18cm]{images/PV à signer.png}%
    \caption{Diagramme de séquences "Signer un PV"}%
\end{figure}

\item \textbf{Diagramme de séquences "Rejeter un PV" :}

Le diagramme de séquences "Rejeter un PV", présenté dans la figure X.X, illustre le processus par lequel un membre DT ou CT refuse un PV en indiquant un motif de rejet.  

\begin{figure}[H]%
    \center%
    \setlength{\fboxsep}{5pt}%
    \setlength{\fboxrule}{0.5pt}%
    \includegraphics[width=17cm,height=11.54cm]{images/rejeter un PV.png}%
    \caption{Diagramme de séquences "Rejeter un PV"}%
\end{figure}

\item \textbf{Diagramme de séquences "Uploader une nouvelle version de PV" :}

Le diagramme de séquences "Uploader une nouvelle version de PV", présenté dans la figure X.X, décrit le processus par lequel le FTE dépose une version corrigée d’un PV après un rejet et le resoumet à la signature des membres.  

\begin{figure}[H]%
    \center%
    \setlength{\fboxsep}{5pt}%
    \setlength{\fboxrule}{0.5pt}%
    \includegraphics[width=17cm,height=18cm]{images/uploader une nouvelle version de pv.png}%
    \caption{Diagramme de séquences "Uploader une nouvelle version de PV"}%
\end{figure}


\end{itemize}

\subsection{Diagrammes de classes de sprint 2}
Un diagramme de classes de deuxième sprint à la figure 4.10 représente visuellement les classes, les interfaces et les relations entre elles au sein d’un système logiciel. Il permet de modéliser la structure statique du système et de décrire les principales entités du système ainsi que leurs relations.  
\begin{figure}[H]%
    \center%
    \setlength{\fboxsep}{5pt}%
    \setlength{\fboxrule}{0.5pt}%
    \includegraphics[width=13cm,height=12 cm]{images/class digram.png}%
    \caption{Diagrammes de classes de sprint 2}%
\end{figure}





































\section{Réalisation}

Cette section présente les interfaces développées au cours de ce sprint.

\item \textbf{Interface de création d'un PV (FTE): }\
En tant que FTE, cette interface permet de créer un nouveau procès-verbal (PV) pour une réunion. Elle est présentée dans la figure 8.

\begin{figure}[H]%
\center%
\setlength{\fboxsep}{5pt}%
\setlength{\fboxrule}{0.5pt}%
\includegraphics[width=16cm,height=9cm]{images/créerpv.png}%
\caption{Interface de création d'un PV}%
\end{figure}

\item \textbf{Interface de consultation d'un PV (FTE, DT, CT): }\
Cette interface permet à tous les utilisateurs de consulter un procès-verbal existant. Elle est présentée dans la figure 9.

\begin{figure}[H]%
\center%
\setlength{\fboxsep}{5pt}%
\setlength{\fboxrule}{0.5pt}%
\includegraphics[width=12cm,height=1cm]{images/consultationpv.png}%
\caption{Interface de consultation d'un PV}%
\end{figure}

\item \textbf{Interface de signature d'un PV (DT ou CT): }\
En tant que Direction Technique ou Commission Technique, cette interface permet de signer électroniquement un PV. Elle est présentée dans la figure 10.

\begin{figure}[H]%
\center%
\setlength{\fboxsep}{5pt}%
\setlength{\fboxrule}{0.5pt}%
\includegraphics[width=16cm,height=6cm]{images/sgintueur.png}%
\caption{Interface de signature d'un PV}%
\end{figure}

\item \textbf{Interface de rejet d'un PV (DT ou CT): }\
En tant que Direction Technique ou Commission Technique, cette interface permet de rejeter un PV avec un commentaire. Elle est présentée dans la figure 11.

\begin{figure}[H]%
\center%
\setlength{\fboxsep}{5pt}%
\setlength{\fboxrule}{0.5pt}%
\includegraphics[width=15cm,height=6cm]{images/REJETPV.png}%
\caption{Interface de rejet d'un PV}%
\end{figure}

\item \textbf{Interface d'upload d'une nouvelle version de PV (FTE): }\
En tant que FTE, cette interface permet d'uploader une nouvelle version d'un PV après un rejet. Elle est présentée dans la figure 12.

\begin{figure}[H]%
\center%
\setlength{\fboxsep}{5pt}%
\setlength{\fboxrule}{0.5pt}%
\includegraphics[width=15cm,height=10cm]{images/nouvelpv.png}%
\caption{Interface d'upload d'une nouvelle version de PV}%
\end{figure}

\end{itemize}


\section{CONCLUSION}

À ce niveau, nous avons terminé le deuxième sprint"Workflow de pv" et réussi à implémenter toutes les fonctionnalités définies lors de la phase de spécification du projet. Nous sommes donc sur le point de conclure notre projet avec succès. 
