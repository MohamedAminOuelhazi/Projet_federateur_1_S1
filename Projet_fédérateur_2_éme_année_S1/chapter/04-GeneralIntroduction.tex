\chapter*{INTRODUCTION GÉNÉRALE}
\markboth{\MakeUppercase{INTRODUCTION GÉNÉRALE}}{}
%\addstarredchapter{GENERAL INTRODUCTION}
\addcontentsline{toc}{chapter}{INTRODUCTION GÉNÉRALE}
\adjustmtc
\thispagestyle{MyStyle}
    \pagenumbering{arabic}% 1 2 3 4 5

Dans un monde numérique en constante évolution, la transformation digitale du secteur de la santé est devenue une nécessité stratégique pour améliorer la qualité des soins et optimiser la gestion administrative. Cette transition vers des systèmes informatisés permet non seulement d'améliorer l'efficacité opérationnelle des cabinets médicaux, mais aussi d'assurer une meilleure traçabilité des dossiers médicaux, de faciliter la communication entre professionnels de santé et patients, et d'intégrer des technologies innovantes telles que l'intelligence artificielle pour améliorer les services de santé.

Les cabinets médicaux traditionnels, confrontés à une croissance significative de leur patientèle et à une complexification des processus de gestion, font face à de nombreux défis : gestion manuelle des rendez-vous source d'erreurs et de conflits, absence de centralisation des dossiers médicaux, difficultés de suivi des factures et paiements, manque d'outils d'aide à la décision pour les patients. Ces problématiques engendrent non seulement une perte de temps considérable pour le personnel médical et administratif, mais représentent également un frein à l'amélioration de la qualité des services offerts aux patients.

C'est pour répondre à ces défis que nous avons conçu et développé une plateforme intégrée de gestion de cabinet médical combinant gestion administrative et intelligence artificielle. Ce projet s'articule autour de plusieurs objectifs majeurs : digitaliser la gestion des rendez-vous et des dossiers patients, automatiser la gestion des factures, intégrer un chatbot IA pour assister les patients dans leurs questions médicales, garantir la sécurité et la confidentialité des données médicales conformément aux normes RGPD, et offrir une interface intuitive accessible sur tous les supports (web, mobile, tablette). La solution technique retenue combine les technologies modernes les plus performantes : Spring Boot pour le backend, Next.js pour le frontend, et l'API OpenAI pour le chatbot intelligent.

Ce rapport présente de manière structurée la méthodologie adoptée, l'analyse approfondie des besoins fonctionnels et non fonctionnels, la conception détaillée de l'architecture système, ainsi que les réalisations techniques et les résultats obtenus pour chaque sprint. Nous y détaillerons comment l'approche agile Scrum a été mise en œuvre pour gérer efficacement ce projet académique, en organisant le travail en trois sprints distincts : gestion de l'espace médecin, gestion de l'espace assistant, et gestion de l'espace patient. À travers ce travail, nous démontrons comment les technologies web modernes et l'intelligence artificielle peuvent transformer radicalement la gestion d'un cabinet médical en une solution digitale performante, sécurisée et centrée sur l'utilisateur.
