\chapter{CHAPITRE 3 : \textbf{Workflow de réunion} }
\begin{spacing}{1.2}
\minitoc
\thispagestyle{MyStyle}
\end{spacing}
\newpage

\section{INTRODUCTION}
Après avoir identifié les besoins fonctionnels et défini les fonctionnalités nécessaires pour notre application web, ce chapitre se concentre sur la description du premier sprint, intitulé "Gestion d'espace de gérant". Nous allons commencer par la spécification fonctionnelle et la conception de ce sprint, en détaillant les fonctionnalités et les interactions du système. Enfin, nous illustrerons ce sprint à l'aide de captures d'écran pour une meilleure compréhension.
\section{Backlog de sprint 1 }
Voici  le tableau \ref{tableau:Backlog du sprint 1} du backlog du sprint 1 : 


\begin{longtable}{|p{3.5cm}|p{9cm}|p{1.5cm}|p{2cm}|}
\caption{Backlog du sprint 1} %

\label{tableau:Backlog du sprint 1}
\\
\hline
 \cellcolor[HTML]{FFF2CC}Fonctionnalités  
& \cellcolor[HTML]{FFF2CC}User Stories   
& \cellcolor[HTML]{FFF2CC}Priorités
& \cellcolor[HTML]{FFF2CC}Estimation\\
\hline
    \multirow{2}{10em}{ Gestion d’Authentification}& En tant que FTE je veux m’authentifier  & Élevée & 3 jours
\\\cline{2-4}
&  En tant que DT je veux m’authentifier  & Élevée & 3 jours
\\\cline{2-4}
&  En tant que CT je veux m’authentifier  & Élevée & 3 jours
\\ \hline
 \multirow{4}{8em}{Gérer Réunion} 
 & En tant que FTE, je veux créer une réunion & Élevée & 5 jours
\\\cline{2-4}
& En tant que FTE, je veux consulter une réunion & Moyenne & 3 jours
\\\cline{2-4}
& En tant que FTE, je veux confirmer une réunion & Moyenne & 3 jours
\\\cline{2-4}
& En tant que FTE, je veux annuler une réunion & Moyenne & 3 jours
\\\cline{2-4}
& En tant que DT, je veux valider une réunion & Moyenne& 1 jours
\\\cline{2-4}
& En tant que DT, je veux rejeter une réunion & Moyenne& 1 jours
\\\cline{2-4}
& En tant que DT, je veux consulter une réunion & Moyenne& 1 jours
\\\cline{2-4}
& En tant que CT, je veux consulter une réunion & Moyenne& 1 jours
\\\hline
\end{longtable}


Chaque user story représente une fonctionnalité ou une action qui liée à réunion qui souhaite pouvoir réaliser dans le système. Les priorités sont classées en élevé, moyenne ou faible , reflétant l'importance relative de chaque user story. 

\section{Spécifications fonctionnelles }
Dans cette partie, nous présentons le digramme de cas d'uilisation du sprint 1 ainsi que les description textuelles. 

\subsection{Diagramme de cas d'utilisation du sprint 1 }
Le diagramme de cas d'utilisation du sprint 1, présenté dans la figure 3.1, illustre les besoins fonctionnels sous la forme d'interactions entre le gérant et le système. 
\begin{figure}[H]%
    \center%
    \setlength{\fboxsep}{5pt}%
    \setlength{\fboxrule}{0.5pt}%
    \includegraphics[width=13cm,height=11cm]{images/sprint1 use case.drawio.png }%

        \caption{Diagramme de cas d'utilisation du sprint 1}%

\end{figure}

\subsection{Description textuelles}
L'objectif de cette activité est de décrire textuellement les scénarios des cas d'utilisation. Il faut préciser comment chaque scénario commence, comment il se termine et comment l'utilisateur interagit avec l'application web. 
\begin{itemize}

\item \textbf{Description textuelle du cas d'utilisation "S'authentifier" :}

Le tableau \ref{tableau:Description textuelle du cas d'utilisation "S'authentifier"} présente la description textuelle du cas d’utilisation "S'authentifier". Ce scénario commence lorsque le utilisateur (FTE, DT, CT )  ouvre l'application et accède à l'écran de connexion. L'utilisateur entre ses identifiants, comprenant un nom d'utilisateur et un mot de passe, dans les champs appropriés. 

\begin{table}[H]
\label{tableau:Description textuelle du cas d'utilisation "S'authentifier"}
\centering
\caption{Description textuelle du cas d'utilisation "S'authentifier" }
\begin{tabular}{|>{\raggedright\arraybackslash}p{5cm}|>{\raggedright\arraybackslash}p{10cm}|}
\hline
 \cellcolor[HTML]{FFF2CC}Cas d'utiliation  
& \cellcolor[HTML]{FFF2CC}S'authentifier   
\\
\hline
 Acteur & User (FTE, DT, CT)
\\ \hline
 Préceondition  & Le système est en service 
\\ \hline
Post-condition & Le user est authentifié et accède à son espace.
\\ \hline
Scénario principal & 
\begin{enumerate}
    \item Le user demande  l'interface de l'authentification.
    \item Le formulaire s'affiche 
    \item le user saisi son nom d'utilisation et mot de passe 
    \item Le système vérifie la saisie.
    \item Le système affiche l’interface correspondante à cet utilisateur.
    
\end{enumerate}
\\ \hline
Scénario alternatif& 
\begin{enumerate} 
    \item Le user saisit des données incomplètes, ce qui entraîne l'affichage d'un message d’erreur par le système.
    \item Le user saisit des données invalides, ce qui entraîne l'affichage d'un message d’erreur par le système.
\end{enumerate}

\\\hline

\end{tabular}
\label{tableau:Description textuelle du cas d'utilisation "S'authentifier"}
\end{table}

\item \textbf{Description textuelle du cas d'utilisation "Créer une réunion" :}
Le tableau \ref{tableau:Description textuelle du cas d'utilisation "Créer une réunion"} présente la description textuelle du cas d’utilisation "Créer une réunion". Ce scénario commence lorsque le FTE accède à l'application et souhaite organiser une nouvelle réunion technique. Le FTE remplit le formulaire de création avec les détails de la réunion, puis le système traite la demande et notifie les acteurs concernés.
\begin{table}[H]
\label{tableau:Description textuelle du cas d'utilisation "Créer une réunion"}
\centering
\caption{Description textuelle du cas d'utilisation "Créer une réunion"}
\begin{tabular}{|>{\raggedright\arraybackslash}p{5cm}|>{\raggedright\arraybackslash}p{10cm}|}
\hline
\cellcolor[HTML]{FFF2CC}Cas d'utilisation  
& \cellcolor[HTML]{FFF2CC}Créer une réunion   
\\
\hline
Acteur & FTE
\\ \hline
Précondition & Le FTE est authentifié et a les droits de création de réunion
\\ \hline
Post-condition & La réunion est créée avec le statut "PENDING" et les notifications sont envoyées à la Direction Technique
\\ \hline
Scénario principal & 
\begin{enumerate}
    \item Le FTE demande l'interface de création de réunion.
    \item Le système affiche le formulaire de création.
    \item Le FTE saisit les informations de la réunion (sujet, description, date, lieu).
    \item Le FTE clique sur "Créer".
    \item Le système vérifie la validité des données saisies.
    \item Le système enregistre la réunion en base de données avec le statut "PENDING".
    \item Le système récupère la liste des membres de la Direction Technique.
    \item Le système génère et envoie des notifications à tous les membres de la Direction Technique.
    \item Le système affiche un message de confirmation au FTE.
\end{enumerate}
\\ \hline
Scénario alternatif& 
\begin{enumerate} 
    \item Le FTE saisit des données incomplètes (champs obligatoires manquants), ce qui entraîne l'affichage de messages d'erreur par le système.
    \item Le FTE saisit une date invalide (passée ou format incorrect), ce qui entraîne l'affichage d'un message d'erreur par le système.
    \item Le système rencontre une erreur technique lors de l'enregistrement, ce qui entraîne l'affichage d'un message d'erreur et une invitation à réessayer.
\end{enumerate}
\\\hline
\end{tabular}
\label{tableau:Description textuelle du cas d'utilisation "Créer une réunion"}
\end{table}

\item \textbf{Description textuelle du cas d'utilisation "Consulter une réunion" :}
Le tableau \ref{tableau:Description textuelle du cas d'utilisation "Consulter une réunion"} présente la description textuelle du cas d’utilisation "Consulter une réunion". Ce scénario commence lorsque l'utilisateur (FTE, DT ou CT) souhaite consulter les détails d'une réunion existante dans le système.
\begin{table}[H]
\label{tableau:Description textuelle du cas d'utilisation "Consulter une réunion"}
\centering
\caption{Description textuelle du cas d'utilisation "Consulter une réunion"}
\begin{tabular}{|>{\raggedright\arraybackslash}p{5cm}|>{\raggedright\arraybackslash}p{10cm}|}
\hline
\cellcolor[HTML]{FFF2CC}Cas d'utilisation  
& \cellcolor[HTML]{FFF2CC}Consulter une réunion   
\\
\hline
Acteur & User (FTE, DT, CT)
\\ \hline
Précondition & L'utilisateur est authentifié et la réunion existe dans le système
\\ \hline
Post-condition & L'utilisateur consulte les détails de la réunion et les documents associés
\\ \hline
Scénario principal & 
\begin{enumerate}
    \item L'utilisateur accède à la liste des réunions.
    \item L'utilisateur sélectionne une réunion dans la liste.
    \item Le système affiche les détails de la réunion (sujet, date, lieu, statut).
    \item Le système affiche la liste des documents associés à la réunion.
    \item L'utilisateur consulte les informations et les documents.
\end{enumerate}
\\ \hline
Scénario alternatif& 
\begin{enumerate} 
    \item La réunion sélectionnée n'existe pas, ce qui entraîne l'affichage d'un message d'erreur "Réunion introuvable".
    \item L'utilisateur n'a pas les permissions pour consulter cette réunion, ce qui entraîne l'affichage d'un message d'erreur "Accès non autorisé".
\end{enumerate}
\\\hline
\end{tabular}
\label{tableau:Description textuelle du cas d'utilisation "Consulter une réunion"}
\end{table}

\item \textbf{Description textuelle du cas d'utilisation "Confirmer une réunion" :}
Le tableau \ref{tableau:Description textuelle du cas d'utilisation "Confirmer une réunion"} présente la description textuelle du cas d’utilisation "Confirmer une réunion". Ce scénario commence lorsque le FTE souhaite confirmer une réunion qui a été préalablement validée par la Direction Technique.
\begin{table}[H]
\label{tableau:Description textuelle du cas d'utilisation "Confirmer une réunion"}
\centering
\caption{Description textuelle du cas d'utilisation "Confirmer une réunion"}
\begin{tabular}{|>{\raggedright\arraybackslash}p{5cm}|>{\raggedright\arraybackslash}p{10cm}|}
\hline
\cellcolor[HTML]{FFF2CC}Cas d'utilisation  
& \cellcolor[HTML]{FFF2CC}Confirmer une réunion   
\\
\hline
Acteur & FTE (Fonctionnaire Technique)
\\ \hline
Précondition & Le FTE est authentifié, la réunion existe et a le statut "SCHEDULED"
\\ \hline
Post-condition & La réunion est confirmée avec le statut "VALIDATED" et les notifications sont envoyées à la Direction Technique et à la Commission Technique
\\ \hline
Scénario principal & 
\begin{enumerate}
    \item Le FTE accède à la liste de ses réunions.
    \item Le FTE sélectionne une réunion avec le statut "SCHEDULED".
    \item Le FTE clique sur le bouton "Confirmer".
    \item Le système affiche une fenêtre de confirmation.
    \item Le FTE confirme l'action.
    \item Le système met à jour le statut de la réunion à "VALIDATED".
    \item Le système envoie des notifications à la Direction Technique et la Commission Technique.
    \item Le système affiche un message de confirmation au FTE.
\end{enumerate}
\\ \hline
Scénario alternatif& 
\begin{enumerate} 
    \item Le FTE annule la confirmation, ce qui entraîne le retour à la liste des réunions sans action.
\end{enumerate}
\\\hline
\end{tabular}
\label{tableau:Description textuelle du cas d'utilisation "Confirmer une réunion"}
\end{table}


\item \textbf{Description textuelle du cas d'utilisation "Annuler une réunion" :}
Le tableau \ref{tableau:Description textuelle du cas d'utilisation "Annuler une réunion"} présente la description textuelle du cas d’utilisation "Annuler une réunion". Ce scénario commence lorsque le FTE souhaite annuler une réunion qu'il a créée.
\begin{table}[H]
\label{tableau:Description textuelle du cas d'utilisation "Annuler une réunion"}
\centering
\caption{Description textuelle du cas d'utilisation "Annuler une réunion"}
\begin{tabular}{|>{\raggedright\arraybackslash}p{5cm}|>{\raggedright\arraybackslash}p{10cm}|}
\hline
\cellcolor[HTML]{FFF2CC}Cas d'utilisation  
& \cellcolor[HTML]{FFF2CC}Annuler une réunion   
\\
\hline
Acteur & FTE (Fonctionnaire Technique)
\\ \hline
Précondition & Le FTE est authentifié, la réunion existe et a un statut annulable ("SCHEDULED" ou "VALIDATED")
\\ \hline
Post-condition & La réunion est annulée avec le statut "CANCELLED" et les notifications sont envoyées à la Direction Technique et à la Commission Technique
\\ \hline
Scénario principal & 
\begin{enumerate}
    \item Le FTE accède à la liste de ses réunions.
    \item Le FTE sélectionne une réunion annulable.
    \item Le FTE clique sur le bouton "Annuler".
    \item Le système affiche une fenêtre de confirmation avec champ de motif optionnel.
    \item Le FTE saisit éventuellement un motif d'annulation.
    \item Le FTE confirme l'annulation.
    \item Le système met à jour le statut de la réunion à "CANCELLED".
    \item Le système enregistre le motif d'annulation si fourni.
    \item Le système envoie des notifications à la Direction Technique et la Commission Technique.
    \item Le système affiche un message de confirmation au FTE.
\end{enumerate}
\\ \hline
Scénario alternatif& 
\begin{enumerate} 
    \item Le FTE annule l'annulation, ce qui entraîne le retour à la liste des réunions sans action.
\end{enumerate}
\\\hline
\end{tabular}
\label{tableau:Description textuelle du cas d'utilisation "Annuler une réunion"}
\end{table}

\item \textbf{Description textuelle du cas d'utilisation "Valider une réunion" :}
Le tableau \ref{tableau:Description textuelle du cas d'utilisation "Valider une réunion"} présente la description textuelle du cas d’utilisation "Valider une réunion". Ce scénario commence lorsque la Direction Technique souhaite valider une réunion en attente de validation technique.
\begin{table}[H]
\label{tableau:Description textuelle du cas d'utilisation "Valider une réunion"}
\centering
\caption{Description textuelle du cas d'utilisation "Valider une réunion"}
\begin{tabular}{|>{\raggedright\arraybackslash}p{4cm}|>{\raggedright\arraybackslash}p{12cm}|}
\hline
\cellcolor[HTML]{FFF2CC}Cas d'utilisation  
& \cellcolor[HTML]{FFF2CC}Valider une réunion   
\\
\hline
Acteur & DT (Direction Technique)
\\ \hline
Précondition & Le DT est authentifié, la réunion existe et a le statut "PENDING"
\\ \hline
Post-condition & La réunion est validée avec le statut "SCHEDULED", les membres CT sont sélectionnés, les fichiers sont associés et les notifications sont envoyées
\\ \hline
Scénario principal & 
\begin{enumerate}
    \item Le DT accède à la liste des réunions en attente de validation.
    \item Le DT sélectionne une réunion et clique sur le bouton "Valider".
    \item Le système affiche le formulaire de validation.
    \item Le DT sélectionne au moins un membre de la Commission Technique.
    \item Le DT upload les fichiers nécessaires.
    \item Le DT saisit un commentaire optionnel.
    \item Le DT clique sur "Confirmer la validation".
    \item Le système met à jour le statut de la réunion à "SCHEDULED".
    \item Le système associe les membres CT sélectionnés, les fichiers uploadés à la réunion.
    \item Le système envoie une notification au FTE et aux membres CT.
    \item Le système affiche un message de confirmation au DT.
\end{enumerate}
\\ \hline
Scénario alternatif& 
\begin{enumerate} 
    \item Le DT n'a pas sélectionné de membre CT ou n'a pas uploadé de fichier, ce qui entraîne l'affichage de messages d'erreur.
    \item Le DT annule la validation, ce qui entraîne le retour à la liste des réunions sans action.

\end{enumerate}
\\\hline
\end{tabular}
\label{tableau:Description textuelle du cas d'utilisation "Valider une réunion"}
\end{table}
\item \textbf{Description textuelle du cas d'utilisation "Rejeter une réunion" :}
Le tableau \ref{tableau:Description textuelle du cas d'utilisation "Rejeter une réunion"} présente la description textuelle du cas d’utilisation "Rejeter une réunion". Ce scénario commence lorsque la Direction Technique souhaite rejeter une réunion en attente de validation technique.
\begin{table}[H]
\label{tableau:Description textuelle du cas d'utilisation "Rejeter une réunion"}
\centering
\caption{Description textuelle du cas d'utilisation "Rejeter une réunion"}
\begin{tabular}{|>{\raggedright\arraybackslash}p{5cm}|>{\raggedright\arraybackslash}p{12cm}|}
\hline
\cellcolor[HTML]{FFF2CC}Cas d'utilisation  
& \cellcolor[HTML]{FFF2CC}Rejeter une réunion   
\\
\hline
Acteur & DT (Direction Technique)
\\ \hline
Précondition & Le DT est authentifié, la réunion existe et a le statut "PENDING"
\\ \hline
Post-condition & La réunion est rejetée avec le statut "REJECTED" et une notification est envoyée au FTE avec le motif du rejet
\\ \hline
Scénario principal & 
\begin{enumerate}
    \item Le DT accède à la liste des réunions en attente de validation.
    \item Le DT clique sur le bouton "Rejeter".
    \item Le système affiche le formulaire de rejet.
    \item Le DT saisit un motif de rejet obligatoire.
    \item Le DT clique sur "Confirmer le rejet".
    \item Le système met à jour le statut de la réunion à "REJECTED".
    \item Le système enregistre le motif du rejet.
    \item Le système envoie une notification au FTE créateur avec le motif du rejet.
    \item Le système affiche un message de confirmation au DT.
\end{enumerate}
\\ \hline
Scénario alternatif& 
\begin{enumerate} 
    \item Le DT n'a pas saisi de motif de rejet, ce qui entraîne l'affichage d'un message d'erreur "Le motif du rejet est obligatoire".
    \item Le DT annule le rejet, ce qui entraîne le retour à la liste des réunions sans action.
\end{enumerate}
\\\hline
\end{tabular}
\label{tableau:Description textuelle du cas d'utilisation "Rejeter une réunion"}
\end{table}


\section{Conception }

Dans le processus de conception, il est crucial d'analyser les différentes interactions et fonctionnalités du système en utilisant des outils et des techniques spécifiques. Nous commencerons par présenter les diagrammes de séquences des cas d'utilisation qui ont été précédemment affinés. Ensuite, nous montrerons le diagramme de classes de  la première sprint. Pour finir, nous présenterons le diagramme d'activité relatif au processus d'authentification.\par

\subsection{Diagrammes de séquences }

Un diagramme de séquences est un diagramme UML (Unified Modeling Language) qui rerésente la succession des messages échangés entre des objets lors de leur communication. Ce type de diagramme comprend un ensemble d'objets, représentés par des lignes de vie, et les messages qu'ils s'envoient au cours de leurs interactions.

\textbf{Digramme de séquences "S'authentifier" }

La figure 3.2  présente le diagramme de séquences détaillé du cas d'utilisation "S'authentifier". Le processus débute lorsque l'utilisateur saisit ses informations dans l'interface de connexion. Si l'utilisateur n'est pas présent dans la base de données, le système affiche un message d'erreur indiquant que les informations de connexion saisies sont incorrectes et invite l'utilisateur à les vérifier. Si l'utilisateur est reconnu,le système vérifie son rôle pour lui permettre d'accéder à l'interface qui lui est associée.  

\begin{figure}[H]%
    \center%
    \setlength{\fboxsep}{5pt}%
    \setlength{\fboxrule}{0.5pt}%
    \includegraphics[width=17cm,height=11.54cm]{images/Diagramme séqaunce authotification.png}%
    \caption{Digramme de séquences "S'authentifier" }%
\end{figure}

\textbf{Diagramme de séquences "Créer une réunion"}\\
Après l'authentification, le FTE accède à son espace pour créer une nouvelle réunion. Il remplit alors le formulaire affiché avec les détails de la réunion, et la réunion est ensuite enregistrée dans la base de données avec le statut "PENDING". Des notifications sont envoyées à la Direction Technique pour validation.
\begin{figure}[H]%
    \center%
    \setlength{\fboxsep}{5pt}%
    \setlength{\fboxrule}{0.5pt}%
    \includegraphics[width=16cm,height=11.43cm]{images/réunion créer.png}%
    \caption{Diagramme de séquences "Créer une réunion"}%
\end{figure}

\textbf{Diagramme de séquences "Consulter une réunion"}\\
Après l'authentification, l'utilisateur (FTE, DT ou CT) accède à la liste des réunions. Il sélectionne une réunion pour consulter ses détails, et le système affiche les informations de la réunion ainsi que les documents associés.
\begin{figure}[H]%
    \center%
    \setlength{\fboxsep}{5pt}%
    \setlength{\fboxrule}{0.5pt}%
    \includegraphics[width=18cm,height=11cm]{images/consulter réunion ( users).png}%
    \caption{Diagramme de séquences "Consulter une réunion"}%
\end{figure}


\textbf{Diagramme de séquences "Confirmer une réunion"}\\
Après l'authentification, le FTE accède à la liste de ses réunions avec le statut "SCHEDULED". Il sélectionne une réunion et clique sur "Confirmer". Le système met à jour le statut en "VALIDATED" et envoie des notifications à la Direction Technique et à la Commission Technique.
\begin{figure}[H]%
    \center%
    \setlength{\fboxsep}{5pt}%
    \setlength{\fboxrule}{0.5pt}%
    \includegraphics[width=18cm,height=19cm]{images/confirmer fte réunion.png}%
    \caption{Diagramme de séquences "Confirmer une réunion"}%
\end{figure}


\textbf{Diagramme de séquences "Annuler une réunion"}\\
Après l'authentification, le FTE accède à la liste de ses réunions. Il sélectionne une réunion annulable et clique sur "Annuler". Après avoir fourni un motif optionnel, le système met à jour le statut en "CANCELLED" et envoie des notifications aux membres concernés.
\begin{figure}[H]%
    \center%
    \setlength{\fboxsep}{5pt}%
    \setlength{\fboxrule}{0.5pt}%
    \includegraphics[width=18cm,height=19cm]{images/annuler fte réunion.png}%
    \caption{Diagramme de séquences "Annuler une réunion"}%
\end{figure}

\textbf{Diagramme de séquences "Valider une réunion"}\\
Après l'authentification, le DT accède à la liste des réunions en attente de validation. Il sélectionne une réunion, remplit le formulaire de validation en sélectionnant les membres CT et en uploadant les fichiers nécessaires, puis valide. Le système met à jour le statut en "SCHEDULED" et envoie des notifications.
\begin{figure}[H]%
    \center%
    \setlength{\fboxsep}{5pt}%
    \setlength{\fboxrule}{0.5pt}%
    \includegraphics[width=18cm,height=19cm]{images/Valider dt réunion.png}%
    \caption{Diagramme de séquences "Valider une réunion"}%
\end{figure}

\textbf{Diagramme de séquences "Rejeter une réunion"}\\
Après l'authentification, le DT accède à la liste des réunions en attente de validation. Il sélectionne une réunion, clique sur "Rejeter", saisit un motif obligatoire, et confirme. Le système met à jour le statut en "REJECTED" et envoie une notification au FTE avec le motif du rejet.
\begin{figure}[H]%
    \center%
    \setlength{\fboxsep}{5pt}%
    \setlength{\fboxrule}{0.5pt}%
    \includegraphics[width=18cm,height=20cm]{images/rejeter dt réunion.png}%
    \caption{Diagramme de séquences "Rejeter une réunion"}%
\end{figure}


\subsection{Diagrammes de classes de sprint 1}

Un diagramme de classes de première sprint à la figure 3.18 représente visuellement les classes, les interfaces et les relations entre elles au sein d’un système logiciel. Il permet de modéliser la structure statique du système et de décrire les principales entités du système ainsi que leurs relations. 

\begin{figure}[H]%
    \center%
    \setlength{\fboxsep}{5pt}%
    \setlength{\fboxrule}{0.5pt}%
    \includegraphics[width=15cm,height=12cm]{images/class digram sprint 1 11.png}
    \caption{Diagrammes de classes de sprint 1}%
\end{figure}
\subsection{Diagramme de activité }
Un diagramme d'activité est une représentation  graphique qui montre le flux d'activités et de décisions dans un processus ou un scénario donné. Il met en évidence les différentes étapes, les actions effectuées et les conditions de transition entre les activités.

\begin{figure}[H]%
    \center%
    \setlength{\fboxsep}{5pt}%
    \setlength{\fboxrule}{0.5pt}%
    \includegraphics[width=6cm,height=9 cm]{images/Diagramme d'activité de authentification .drawio.png}%
    \caption{Diagrammes  d'activité "S'authentifier"}%
\end{figure}





















\section{Réalisation}

Cette section présente les interfaces développées au cours de ce sprint.

\begin{itemize}
\item \textbf{Interface d'authentification des utilisateurs (FTE, DT, CT): }\
L'interface d'authentification permet aux utilisateurs (FTE, Direction Technique, Commission Technique) de se connecter en entrant leur nom d'utilisateur et leur mot de passe. Elle est présentée dans la figure 1.

\begin{figure}[H]%
\center%
\setlength{\fboxsep}{5pt}%
\setlength{\fboxrule}{0.5pt}%
\includegraphics[width=16cm,height=6cm]{images/auth.png}%
\caption{Interface d'authentification des utilisateurs}%
\end{figure}

\item \textbf{Interface de création d'une réunion (FTE): }\
En tant que FTE, cette interface permet de créer une nouvelle réunion en spécifiant les détails nécessaires. Elle est présentée dans la figure 2.

\begin{figure}[H]%
\center%
\setlength{\fboxsep}{5pt}%
\setlength{\fboxrule}{0.5pt}%
\includegraphics[width=16cm,height=8cm]{images/créerréunion.png}%
\caption{Interface de création d'une réunion}%
\end{figure}

\item \textbf{Interface de consultation d'une réunion (FTE, DT, CT): }\
Cette interface permet à tous les utilisateurs de consulter les détails d'une réunion existante. Elle est présentée dans la figure 3.

\begin{figure}[H]%
\center%
\setlength{\fboxsep}{5pt}%
\setlength{\fboxrule}{0.5pt}%
    \centering
\includegraphics[width=16cm,height=8cm]{images/consulterréunion.png}%
\caption{Interface de consultation d'une réunion}%
\end{figure}

\item \textbf{Interface de confirmation d'une réunion (FTE): }\
En tant que FTE, cette interface permet de confirmer définitivement une réunion validée. Elle est présentée dans la figure 4.

\begin{figure}[H]%
\center%
\setlength{\fboxsep}{5pt}%
\setlength{\fboxrule}{0.5pt}%
\includegraphics[width=16cm,height=1cm]{images/confirmer.png}%
\caption{Interface de confirmation d'une réunion}%
\end{figure}

\item \textbf{Interface d'annulation d'une réunion (FTE): }\
En tant que FTE, cette interface permet d'annuler une réunion planifiée. Elle est présentée dans la figure 5.

\begin{figure}[H]%
\center%
\setlength{\fboxsep}{5pt}%
\setlength{\fboxrule}{0.5pt}%
\includegraphics[width=13cm,height=6cm]{images/annuler.png}%
\caption{Interface d'annulation d'une réunion}%
\end{figure}

\item \textbf{Interface de validation d'une réunion (DT): }\
En tant que Direction Technique, cette interface permet de valider une réunion proposée, d'uploader des fichiers et de sélectionner les membres de la Commission Technique. Elle est présentée dans la figure 6.

\begin{figure}[H]%
\center%
\setlength{\fboxsep}{5pt}%
\setlength{\fboxrule}{0.5pt}%
\includegraphics[width=13cm,height=12cm]{images/validationréunion.png}
\caption{Interface de validation d'une réunion}%
\end{figure}

\item \textbf{Interface de rejet d'une réunion (DT): }\
En tant que Direction Technique, cette interface permet de rejeter une réunion proposée avec des motifs. Elle est présentée dans la figure 7.

\begin{figure}[H]%
\center%
\setlength{\fboxsep}{5pt}%
\setlength{\fboxrule}{0.5pt}%
    \includegraphics[width=0.5\linewidth]{images/rejetréunion.png}
\caption{Interface de rejet d'une réunion}%
\end{figure}


\end{itemize}


\section{CONCLUSION}
Suite à une analyse approfondie du premier sprint "Workflow de réunion", nous sommes en mesure de poursuivre notre exploration avec une compréhension claire et complète de l'objectif recherché, ce qui nous permet de passer au chapitre suivant. 

