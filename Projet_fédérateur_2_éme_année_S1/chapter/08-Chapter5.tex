\chapter{CHAPITRE 5 : \textbf{Phase de clôture } }
\begin{spacing}{1.2}
\minitoc
\thispagestyle{MyStyle}
\end{spacing}
\newpage

\section{INTRODUCTION}
Ce dernier chapitre de notre projet est dédié à la phase de clôture. Nous y présenterons les technologies matérielles et logicielles utilisées, ainsi que les langages de programmation employés pour la réalisation du projet. De plus, nous mettrons en avant les interfaces importantes de notre application web.  

\section{Environnement de développement}
Dans cette partie, nous allons présenter l’environnements de travail matériel et logiciel utilisé dans la réalisation de notre projet. 

\subsection{Environnement matériel   }

Pendant la réalisation de notre application, j'ai utilisé des équipement ayant les caractéristiques présentées dans le tableau 
\ref{tableau:Description des équipements de développement}


\begin{table}[H]
\label{tableau:Description des équipements de développement}
\centering
\caption{Description des équipements de développement}
\begin{tabular}{|p{5cm}|p{5cm}|}
\hline
 \cellcolor[HTML]{FFF2CC}Caractéristique   
& \cellcolor[HTML]{FFF2CC}Ordinateur  
\\
\hline
Modéle & 
MSI \\
\hline
 Mémoire RAM  & 
24 \\
\hline
 Systéme d'exploitation & 
Microsoft Windows 10 Professionnel \\

\hline
 Carte Ghraphique  & 
NVIDIA GeForce GTX 1650 Ti  \\

\hline
Processeur  & 
10th Gen Intel(R) Core(TM) i7-10750H  \\

\hline
 
\end{tabular}
\label{tableau:Description des équipements de développement}
\end{table}

\subsection{Environnement logiciel }

Dans cette partie nous allons présenter l'environnement logiciel utilisé dans le développement de notre application. 

\subsubsection{Outils de développement}

Commençons  par les outils de développement adoptés : 

\begin{itemize}
    \item \textbf{Angular : }Angular est un framework open source  développé par Google. Ce framework est utilisé pour développer des applications web et mobile. Avec cette technologie, on réalise des interfaces de type monopage ou “one page” qui fonctionnent sans rechargement de la page web.Angular propose un ensemble de conventions et d’outils pour délimiter les bases d’une solution. Les développements sont ainsi optimisés, s’effectuent plus rapidement et de manière plus sûre.Le Framework est basé sur une architecture du type MVC et permet donc de séparer les données, le visuel et les actions pour une meilleure gestion des responsabilités. Un type d’architecture qui a largement fait ses preuves et qui permet une forte maintenabilité et une amélioration du travail collaboratif. \cite{titleBIB10}

 

\begin{figure}[H]%
    \center%
    \setlength{\fboxsep}{5pt}%
    \setlength{\fboxrule}{0.5pt}%
    \includegraphics[width=3 cm,height=3cm]{images/angular.png}%
    
    \caption{Logo Angular }%
    
\end{figure}
  \item \textbf{TypeScript : }TypeScript est un langage de programmation développé par Microsoft en 2012. Son ambition principale est d’améliorer la productivité de développement d’applications complexes. C’est un langage open source, développé comme un sur-ensemble de Javascript. Ce qu’il faut comprendre, c’est que tout code valide en Javascript l’est également en TypeScript. Cependant, ce langage introduit des fonctionnalités optionnelles comme le typage ou la programmation orientée objet. Pour bénéficier de ces fonctionnalités, aucune librairie n’est requise. Il suffit d’utiliser l’outil de compilation de TypeScript pour le transpiler (compiler le code source d’un langage en un autre langage) en Javascript. Ainsi, le code exécuté sera un équivalent Javascript du code TypeScript compilé. \cite{titleBIB11}
    \begin{figure}[H]%
    \center%
    \setlength{\fboxsep}{5pt}%
    \setlength{\fboxrule}{0.5pt}%
    \includegraphics[width=2 cm,height=2cm]{images/ts-logo-512.png}%
    
    \caption{Logo TypeScript }%
    
\end{figure}
    \item   \textbf{Spring Boot: }Le Spring Framework est une infrastructure open source d'entreprise largement utilisée pour créer des applications autonomes de production fonctionnant sur la machine virtuelle Java (JVM). Spring Boot, un outil associé, accélère et simplifie le développement d'applications Web et de microservices avec Spring Framework en offrant trois fonctionnalités principales : la configuration automatique, une approche directive de la configuration et la possibilité de créer des applications autonomes. Ces fonctionnalités combinées permettent de configurer et d'installer une application Spring avec un minimum d'efforts.\cite{titleBIB12}
    \begin{figure}[H]%
    \center%
    \setlength{\fboxsep}{5pt}%
    \setlength{\fboxrule}{0.5pt}%
    \includegraphics[width=6cm,height=2cm]{images/Design sans titre (1).jpg}%
    
    \caption{Logo Spring Boot }%
    
\end{figure}
  \item \textbf{Boostrape : }Bootstrap est un framework développé par l’équipe du réseau social Twitter. Proposé en open source (sous licence MIT), ce framework utilisant les langages HTML, CSS et JavaScript fournit aux développeurs des outils pour créer un site facilement. Ce framework est pensé pour développer des sites avec un design responsive, qui s’adapte à tout type d’écran, et en priorité pour les smartphones. Il fournit des outils avec des styles déjà en place pour des typographies, des boutons, des interfaces de navigation et bien d’autres encore. On appelle ce type de framework un "Front-End Framework".\cite{titleBIB13}
    \begin{figure}[H]%
    \center%
    \setlength{\fboxsep}{5pt}%
    \setlength{\fboxrule}{0.5pt}%
    \includegraphics[width=3cm,height=3cm]{images/icons8-bootstrap-240.png}%
    
    \caption{Logo Boostrape }%
    
\end{figure}
\item \textbf{nodejs : }Node.js est un environnement d’exécution single-thread, open-source et multiplateforme permettant de créer des applications rapides et évolutives côté serveur et en réseau. Il fonctionne avec le moteur d’exécution JavaScript V8 et utilise une architecture d’E/S non bloquante et pilotée par les événements, ce qui le rend efficace et adapté aux applications en temps réel. \cite{titleBIB14}
  \begin{figure}[H]%
    \center%
    \setlength{\fboxsep}{5pt}%
    \setlength{\fboxrule}{0.5pt}%
    \includegraphics[width=5cm,height=3cm]{images/pngwing.com.png}%
    
    \caption{Logo Node JS }%
    
\end{figure}

\item \textbf{HTML 5 : }HTML est l’abréviation de « hypertext markup language » (langage de balisage hypertexte) et est un langage relativement simple utilisé pour créer des pages web. Comme il n’autorise pas les variables ou les fonctions, il n’est pas considéré comme un « langage de programmation », mais plutôt comme un « langage de balisage », c’est-à-dire un langage qui utilise des balises pour définir les éléments d’un document. Le HTML5, pour HyperText Markup Language 5, est une version du célèbre format HTML utilisé pour concevoir les sites Internet. Celui-ci se résume à un langage de balisage qui sert à l’écriture de l’hypertexte indispensable à la mise en forme d’une page Web. \cite{titleBIB15}
  \begin{figure}[H]%
    \center%
    \setlength{\fboxsep}{5pt}%
    \setlength{\fboxrule}{0.5pt}%
    \includegraphics[width=3cm,height=3cm]{images/icons8-html-5-240.png}%
    
    \caption{Logo HTML 5 }%
    
\end{figure}


\item \textbf{CSS : }Cascading Style Sheets (CSS) est un langage de programmation qui vous permet de déterminer la design des documents électroniques. À l’aide de simples instructions, présentées dans des codes sources clairs, les éléments de la page Web comme la mise en page, la couleur et la police peuvent ainsi être modulés à souhait. Grâce aux feuilles de style en cascade, la structure sémantique et le contenu du document restent totalement intacts. CSS a été lancé au milieu des années 90 et est à présent considéré comme le langage de feuilles de style standard sur le World Wide Web.  \cite{titleBIB16}
  \begin{figure}[H]%
    \center%
    \setlength{\fboxsep}{5pt}%
    \setlength{\fboxrule}{0.5pt}%
    \includegraphics[width=3cm,height=3cm]{images/icons8-css3-240.png}%
    
    \caption{Logo CSS }%
    
\end{figure}

\item \textbf{MySQL : }MySQL est un système de gestion de bases de données relationnelles SQL open source développé et supporté par Oracle. C’est la réponse courte, en une phrase, à la question « qu’est-ce que MySQL », mais décomposons cela en termes un peu plus humains. Une base de données n’est qu’une collection structurée de données qui est organisée pour en faciliter l’utilisation et la récupération. Pour un site WordPress, ces « données » sont des choses comme le texte de vos articles de blog, des informations pour tous les utilisateurs enregistrés sur votre site, des données chargées automatiquement, des configurations de paramètres importants, etc. MySQL n’est qu’un système populaire qui peut stocker et gérer ces données pour vous, et c’est une solution de base de données particulièrement populaire pour les sites WordPress. \cite{titleBIB17}
  \begin{figure}[H]%
    \center%
    \setlength{\fboxsep}{5pt}%
    \setlength{\fboxrule}{0.5pt}%
    \includegraphics[width=5cm,height=3cm]{images/logo-mysql-170x115.png}%
    
    \caption{Logo MySQL }%
    
\end{figure}
\end{itemize}


\subsubsection{Les Logiciels de développement }

Commençons  par les outils de développement adoptés : 

\begin{itemize}
    \item \textbf{VS Code : }Visual Studio Code est un éditeur de code gratuit conçu pour aider les programmeurs à écrire, déboguer et corriger du code grâce à la fonctionnalité IntelliSense. Il simplifie l'écriture de code pour les utilisateurs. Bien que certains le considèrent comme un compromis entre un IDE complet et un simple éditeur de texte, la décision finale revient aux codeurs.\cite{titleBIB18}
\begin{figure}[H]%
    \center%
    \setlength{\fboxsep}{5pt}%
    \setlength{\fboxrule}{0.5pt}%
    \includegraphics[width=3 cm,height=3cm]{images/vscode.png}%
    
    \caption{Logo VS Code }%
    
\end{figure}  
  \item \textbf{Postman  : }Postman est un logiciel conçu pour créer et tester des requêtes HTTP. Grâce à son interface ergonomique et intuitive, il permet une personnalisation minutieuse des requêtes. Vous pouvez choisir la méthode de la requête, entrer l’URL du serveur à interroger, et ajouter tous les paramètres nécessaires pour une requête HTTP. Le logiciel conserve un historique de vos requêtes, ce qui le rend particulièrement utile pour tester une API. Nous l'utiliserons pour nous initier au protocole HTTP en créant diverses requêtes. \cite{titleBIB19}
    \begin{figure}[H]%
    \center%
    \setlength{\fboxsep}{5pt}%
    \setlength{\fboxrule}{0.5pt}%
    \includegraphics[width=6cm,height=2cm]{images/Postman_(software).png}%
    
    \caption{Logo Postman }%
    
\end{figure}
    \item   \textbf{XAMPP: }XAMPP est un ensemble de logiciels permettant de configurer un serveur Web local, un serveur FTP et un serveur de messagerie électronique. Cette distribution de logiciels libres (comprenant Apache, MariaDB, Perl, et PHP) est réputée pour sa souplesse d'utilisation et son installation simple et rapide. \cite{titleBIB20}
    \begin{figure}[H]%
    \center%
    \setlength{\fboxsep}{5pt}%
    \setlength{\fboxrule}{0.5pt}%
    \includegraphics[width=4cm,height=4cm]{images/58482973cef1014c0b5e49fd.png}%
    
    \caption{Logo XAMPP }%
    
\end{figure}
\end{itemize}


\section{CONCLUSION}

Dans ce chapitre, nous avons exploré l’environnement de travail utilisé pour la mise en œuvre de notre projet. En conclusion, nous présenterons une synthèse générale englobant l'ensemble de nos réalisations.

 