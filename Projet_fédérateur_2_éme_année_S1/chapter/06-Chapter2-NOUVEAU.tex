\chapter{CHAPITRE 2 : Étude préliminaire}
\begin{spacing}{1.2}
\minitoc
\thispagestyle{MyStyle}
\end{spacing}
\newpage


\section{INTRODUCTION}
Ce chapitre présente l'étude préliminaire menée dans le cadre du développement de la plateforme de gestion de cabinet médical intégrant l'intelligence artificielle. L'objectif de cette étude est de spécifier de manière exhaustive les besoins fonctionnels et non fonctionnels du système, de modéliser les processus métiers à travers des diagrammes UML, et de planifier la mise en œuvre en adoptant une méthodologie agile Scrum structurée en trois sprints distincts. Cette phase d'analyse constitue le fondement technique et fonctionnel du projet, permettant d'établir un socle solide pour la conception et la réalisation de la solution, tout en garantissant l'adéquation entre les attentes des acteurs médicaux (médecins, assistants, patients) et les livrables attendus.


\section{Spécification des besoins}
La phase de spécification des besoins est une étape fondamentale qui vise à acquérir une compréhension approfondie du contexte du système de gestion de cabinet médical. Elle contient l'identification des acteurs, la définition des besoins fonctionnels et non fonctionnels, ainsi que la détermination du diagramme de cas d'utilisation global et du diagramme de classes global. 

\subsection{Identification des acteurs}

L'analyse des processus métiers a permis d'identifier quatre acteurs principaux interagissant avec le système de gestion de cabinet médical :


\renewcommand{\labelitemi}{\tiny$\bullet$}

\begin{itemize}[leftmargin=2cm, topsep=0pt]
        \begin{spacing}{1.25}
        \item \textbf{Patient} : Utilisateur externe qui consulte et gère ses informations médicales. Il peut prendre des rendez-vous en ligne (en sélectionnant une date et un créneau disponible), consulter ses rendez-vous, consulter ses dossiers médicaux et documents (ordonnances, analyses), télécharger ses fichiers médicaux, et interagir avec le chatbot IA pour poser des questions médicales. Le patient s'inscrit avec une vérification par email pour garantir l'authenticité de son compte.
        
        \item \textbf{Médecin} : Professionnel de santé principal qui supervise l'ensemble du cabinet. Il peut gérer les assistants (création, activation/désactivation), consulter le calendrier des rendez-vous, créer et modifier les dossiers médicaux avec upload de documents (ordonnances, résultats d'analyses), gérer les factures et enregistrer les paiements, et générer des rapports financiers. Il a un accès complet à toutes les fonctionnalités du système.
        
        \item \textbf{Assistant} : Personnel administratif du cabinet gérant les aspects organisationnels. Il peut planifier et modifier des rendez-vous, créer des factures (uniquement pour les patients liés via les rendez-vous qu'il a créés). Ses accès sont restreints pour garantir la confidentialité et le contrôle d'accès basé sur les rôles. 
        
        \item \textbf{Système (Chatbot IA)} : Assistant virtuel intelligent réservé aux patients. Il analyse les questions médicales posées par les patients, fournit des conseils généraux basés sur l'intelligence artificielle (OpenAI), et enregistre l'historique des conversations. Le chatbot n'établit pas de diagnostic mais oriente les patients vers une consultation si nécessaire.
        \end{spacing}
\end{itemize}
\begin{spacing}{0.5}
\end{spacing}

\subsection{Les besoins fonctionnels}
Les fonctionnalités de notre application web et les services qu'elle offre aux différents acteurs sont présentés ci-dessous : 

\renewcommand{\labelitemi}{\tiny$\bullet$}

\textbf{1- }\textbf{En tant que Patient : }

\begin{itemize}[leftmargin=2cm, topsep=0pt]
        \begin{spacing}{1.25}
        \item \textbf{S'inscrire} : Le patient peut créer un compte avec vérification par email (code à 6 chiffres)
        \item \textbf{S'authentifier} : Le patient a la possibilité de se connecter à son compte de manière sécurisée avec username et mot de passe (JWT).
        \item \textbf{Prendre un rendez-vous} : Le patient peut créer un rendez-vous en ligne en sélectionnant une date dans le calendrier, en consultant les créneaux disponibles du médecin, et en choisissant un créneau libre (les dates passées sont désactivées).
        \item \textbf{Consulter mes rendez-vous} : Le patient peut visualiser ses prochains rendez-vous (date, heure, motif, médecin, statut) et l'historique complet.
        \item \textbf{Consulter mes dossiers médicaux} : Le patient peut accéder à ses dossiers médicaux complets (diagnostic, traitement, symptômes, observations, recommandations).
        \item \textbf{Télécharger mes documents} : Le patient peut télécharger ses ordonnances, résultats d'analyses et autres documents médicaux.
        \item \textbf{Consulter mes factures} : Le patient peut visualiser toutes ses factures avec statut (payées et impayées), montant, date d'émission.
        \item \textbf{Poser des questions au chatbot IA} : Le patient peut interagir avec l'assistant virtuel intelligent (réservé aux patients) pour obtenir des conseils médicaux généraux, poser des questions sur ses consultations passées avec mention de rendez-vous spécifique (@RDV : date).
        \item \textbf{Gérer mes notifications} : Le patient peut consulter ses notifications (nouveaux RDV, dossiers, factures), marquer comme lues.
        \item \textbf{Consulter le tableau de bord} : Le patient peut visualiser son espace personnel avec accueil personnalisé, prochains rendez-vous (date, médecin, motif), et accès rapides (Mes rendez-vous, Mes dossiers médicaux).
        \item \textbf{Gérer son profil} : Le patient peut consulter et modifier ses informations personnelles (nom, prénom, email, téléphone, date de naissance, adresse).
        \item \textbf{Gérer les paramètres} : Le patient peut configurer ses préférences de notifications (activation/désactivation emails, délai de rappel en heures, email personnalisé pour notifications), changer son mot de passe, et supprimer son compte.
        \end{spacing}
\end{itemize}
\begin{spacing}{0.5}
\end{spacing}


\renewcommand{\labelitemi}{\tiny$\bullet$}

\textbf{2- }\textbf{En tant que Médecin :}

\begin{itemize}[leftmargin=2cm, topsep=0pt]
        \begin{spacing}{1.25}
        \item \textbf{S'authentifier} : Le médecin peut se connecter à son compte sécurisé.
        \item \textbf{Créer un assistant} : Le médecin peut ajouter de nouveaux assistants au cabinet avec attribution de comptes.
        \item \textbf{Gérer les assistants} : Le médecin peut consulter, modifier les informations, activer/désactiver ou supprimer définitivement des assistants.
        \item \textbf{Gérer les patients} : Le médecin peut consulter la liste complète des patients, modifier leurs informations personnelles, ou supprimer un patient du système.
        \item \textbf{Consulter les rendez-vous} : Le médecin peut visualiser tous les rendez-vous, incluant les rendez-vous créés par les assistants et les patients.
        \item \textbf{Gérer les rendez-vous} : Le médecin peut créer des rendez-vous pour les patients, consulter les détails, vérifier les créneaux disponibles et consulter l'historique complet.
        \item \textbf{Modifier un dossier médical} : Le médecin peut mettre à jour les informations d'un dossier existant (modification du diagnostic, traitement, ajout d'observations).
        \item \textbf{Ajouter des documents} : Le médecin peut uploader des ordonnances, résultats d'analyses (PDF, images) dans les dossiers patients.
        \item \textbf{Télécharger des documents} : Le médecin peut télécharger les documents médicaux associés à un dossier.
        \item \textbf{Créer des factures} : Le médecin peut créer des factures pour n'importe quel patient du cabinet.
        \item \textbf{Enregistrer les paiements} : Le médecin peut enregistrer les paiements reçus avec différents modes (CB, espèces, chèque, virement).
        \item \textbf{Supprimer une facture} : Le médecin peut supprimer définitivement une facture (réservé au médecin uniquement).
        \item \textbf{Générer des rapports financiers} : Le médecin peut obtenir des statistiques détaillées sur une période donnée (revenus, factures payées/impayées, répartition par mode de paiement). Cette fonctionnalité est exclusive au médecin avec redirection automatique si un assistant ou patient tente d'y accéder.
        \item \textbf{Consulter le tableau de bord} : Le médecin peut visualiser les statistiques en temps réel (nombre de patients, rendez-vous du mois, factures en attente) et les prochains rendez-vous.
        \item \textbf{Gérer son profil} : Le médecin peut consulter et modifier ses informations personnelles (nom, prénom, email, téléphone, date de naissance, spécialité, description).
        \item \textbf{Gérer les paramètres} : Le médecin peut configurer ses préférences de notifications (activation/désactivation emails, délai de rappel, email personnalisé), changer son mot de passe et supprimer son compte.
        \end{spacing}
\end{itemize}
\begin{spacing}{0.5}
\end{spacing}

\renewcommand{\labelitemi}{\tiny$\bullet$}

\textbf{3- }\textbf{En tant qu'Assistant :}

\begin{itemize}[leftmargin=2cm, topsep=0pt]
        \begin{spacing}{1.25}
        \item \textbf{S'authentifier} : L'assistant peut se connecter avec ses identifiants (username et mot de passe) fournis par le médecin.
        \item \textbf{Consulter les patients} : L'assistant peut voir la liste complète des patients ou uniquement ses patients liés via les rendez-vous qu'il a créés.
        \item \textbf{Modifier un patient} : L'assistant peut mettre à jour les informations personnelles d'un patient.
        \item \textbf{Créer un rendez-vous} : L'assistant peut planifier des rendez-vous en sélectionnant patient, médecin, date et créneau horaire.
        \item \textbf{Consulter les créneaux disponibles} : L'assistant peut vérifier les disponibilités du médecin avant de fixer un rendez-vous.
        \item \textbf{Consulter les rendez-vous} : L'assistant peut visualiser tous ses rendez-vous créés, les rendez-vous du jour et l'historique complet.
        \item \textbf{Modifier un rendez-vous} : L'assistant peut modifier les rendez-vous qu'il a créés (date, heure, motif).
        \item \textbf{Annuler un rendez-vous} : L'assistant peut annuler un rendez-vous qu'il a créé .
        \item \textbf{Créer une facture} : L'assistant peut créer des factures UNIQUEMENT pour les patients liés aux rendez-vous qu'il a créés .
        \item \textbf{Consulter les factures} : L'assistant visualise uniquement les factures liées aux rendez-vous qu'il a créés (filtrage automatique par le backend).
        \item \textbf{Enregistrer les paiements} : L'assistant peut marquer une facture comme payée avec sélection du mode de paiement.
        \item \textbf{Consulter le tableau de bord} : L'assistant peut visualiser ses statistiques personnelles (nombre de patients gérés, rendez-vous créés, factures du mois, rendez-vous du jour) .
        \item \textbf{Gérer son profil} : L'assistant peut consulter et modifier ses informations personnelles (nom, prénom, email, téléphone).
        \item \textbf{Gérer les paramètres} : L'assistant peut configurer ses préférences de notifications (activation/désactivation emails, délai de rappel, email personnalisé), changer son mot de passe et supprimer son compte.
        \end{spacing}
\end{itemize}

\subsection{Les besoins non fonctionnels}
\renewcommand{\labelitemi}{\tiny$\bullet$}

Les exigences non fonctionnelles définissent les objectifs relatifs à la performance, la sécurité et la qualité du système de gestion de cabinet médical. Ces aspects, bien qu'invisibles pour l'utilisateur, sont cruciaux pour garantir un système robuste, fiable et conforme aux régulations médicales. Notre application doit satisfaire aux critères suivants :

\begin{itemize}[leftmargin=2cm, topsep=0pt]
        \begin{spacing}{1.25}
        \item \textbf{Sécurité :} L'application doit garantir la confidentialité et l'intégrité des données médicales sensibles. Cela inclut l'authentification sécurisée par JWT (JSON Web Tokens), le chiffrement des mots de passe avec BCrypt, le contrôle d'accès basé sur les rôles (RBAC) pour restreindre les accès selon le type d'utilisateur, et la conformité au RGPD pour la protection des données personnelles et médicales. Le système doit également tracer tous les accès aux dossiers médicaux.
        
        \item \textbf{Fiabilité :} L'application web doit être fiable et robuste, minimisant les erreurs système. Elle doit assurer la cohérence des données (pas de conflits de rendez-vous, intégrité référentielle en base de données), gérer les transactions de manière atomique (création de dossier + upload de fichiers), et implémenter un système de sauvegarde automatique quotidienne des données.
        
        \item \textbf{Utilisabilité :} L'interface doit être intuitive et accessible même pour des utilisateurs non techniques. Navigation claire avec menus organisés par rôle, formulaires avec validation en temps réel et messages d'erreur explicites, design responsive compatible mobile/tablette/desktop, et support multi-navigateurs (Chrome, Firefox, Safari, Edge).
        
        \item \textbf{Maintenabilité :} Le code doit être modulaire et documenté pour faciliter l'évolution. Architecture en couches (Controllers, Services, Repositories), respect des principes SOLID, tests unitaires et d'intégration, et documentation technique complète (JavaDoc, commentaires).
        
        \item \textbf{Évolutivité :} Le système doit permettre l'ajout facile de nouvelles fonctionnalités (téléconsultation, gestion de stock de médicaments, intégration avec d'autres systèmes médicaux). Architecture modulaire permettant l'ajout de nouveaux modules sans refonte complète.
        
        \item \textbf{Conformité réglementaire :} Respect du secret médical, traçabilité complète des accès et modifications, archivage légal des dossiers médicaux, et conformité RGPD avec gestion du consentement et droit à l'oubli.
        \end{spacing}
\end{itemize}
\begin{spacing}{0.5}
\end{spacing}


\section{Détails fonctionnels}

\renewcommand{\labelitemi}{\tiny$\bullet$}
Dans cette section, nous présenterons le diagramme de cas d'utilisation général et le diagramme de classes global pour le système de gestion de cabinet médical.

\subsection{Diagramme de cas d'utilisation global}
Un diagramme de cas d'utilisation global est un outil de modélisation UML qui offre une vue d'ensemble des interactions entre les utilisateurs (appelés acteurs) et le système. Il représente visuellement les fonctionnalités du système et la manière dont les quatre acteurs (Patient, Médecin, Assistant, Chatbot IA) les exploitent. \par

\begin{figure}[H]%
    \center%
    \setlength{\fboxsep}{5pt}%
    \setlength{\fboxrule}{0.5pt}%
    
    \includegraphics[width=18 cm,height=18cm]{images/usecasecabinet.jpg}%
    
    \caption{Diagramme de cas d'utilisation global}%
    
\end{figure}

\subsection{Diagramme de classes global}

Un diagramme de classes est essentiel pour visualiser la structure du système de gestion de cabinet médical. Il identifie les entités principales (User, Patient, Médecin, Assistant, RendezVous, DossierPatient, Facture, Document, Notification), ainsi que leurs attributs, méthodes et relations (héritage, associations, compositions). Il facilite la communication entre les membres de l'équipe de développement et sert de fondation pour la conception et l'implémentation du système.   \par

\begin{figure}[H]%
    \center%
    \setlength{\fboxsep}{5pt}%
    \setlength{\fboxrule}{0.5pt}%
    \includegraphics[width=15 cm,height=25cm]{images/class-diagram-cabinet-medical.png}%
    
    \caption{Diagramme de classes global}%
    
\end{figure}

\section{Mise en œuvre}
Dans cette partie nous allons présenter le Product backlog, la planification des sprints, le diagramme de Gantt et l'architecture du système. 

\subsection{Product backlog}
\renewcommand{\labelitemi}{\tiny$\bullet$}
Comme nous l'avons indiqué dans le premier chapitre, le backlog produit est un élément essentiel qui décrit les besoins et les fonctionnalités attendues, classés par ordre de priorité. Chaque élément est détaillé sous forme de user story. Le tableau suivant présente le backlog produit de notre solution de gestion de cabinet médical, contenant les champs suivants : \\
\begin{itemize}[leftmargin=2cm, topsep=0pt]
        \begin{spacing}{1.25}
  
        \item  \textbf{Fonctionnalités  : }Il s'agit d'un résumé bref de l'histoire utilisateur
        \item  \textbf{\textbf{User Stories}  :} Caractérise la fonctionnalité désirée par l'utilisateur 
        \item  \textbf{Priorité  : }Caractérise l'importance de la fonctionnalité 
        \item  \textbf{Estimation : } Une évaluation du temps nécessaire pour réaliser chaque user story  
\\
        \end{spacing}
\end{itemize}
\begin{spacing}{0.5}
\end{spacing}

\begin{longtable}{|p{3.5cm}|p{9cm}|p{1.5cm}|p{2cm}|}
\caption{Backlog Produit} %

\label{tableau:Backlog Produit}
\\
\hline
 \cellcolor[HTML]{FFF2CC}Fonctionnalités  
& \cellcolor[HTML]{FFF2CC}User Stories   
& \cellcolor[HTML]{FFF2CC}Priorités
& \cellcolor[HTML]{FFF2CC}Estimation\\
\hline
\multirow{3}{10em}{Gestion d'Authentification}
& En tant que Médecin, je veux m'authentifier  & Élevée & 2 jours
\\\cline{2-4}
&  En tant qu'Assistant, je veux m'authentifier  & Élevée & 2 jours
\\\cline{2-4}
&  En tant que Patient, je veux m'inscrire avec vérification email  & Élevée & 4 jours
\\\cline{2-4}
&  En tant que Patient, je veux m'authentifier  & Élevée & 2 jours
\\ \hline
\multirow{5}{8em}{Gérer Assistants (Médecin)} 
& En tant que Médecin, je veux créer un assistant & Élevée & 3 jours
\\\cline{2-4}
& En tant que Médecin, je veux consulter la liste des assistants & Moyenne & 2 jours
\\\cline{2-4}
& En tant que Médecin, je veux modifier un assistant & Moyenne & 2 jours
\\\cline{2-4}
& En tant que Médecin, je veux activer/désactiver un assistant & Moyenne & 2 jours
\\\cline{2-4}
& En tant que Médecin, je veux supprimer définitivement un assistant & Basse & 1 jour
\\ \hline
\multirow{5}{8em}{Gérer Patients}
& En tant que Patient, je veux m'inscrire via /register (page publique) & Élevée & 3 jours
\\\cline{2-4}
& En tant que Médecin/Assistant, je veux consulter tous les patients & Moyenne & 2 jours
\\\cline{2-4}
& En tant qu'Assistant, je veux consulter mes patients liés (via RDV créés) & Moyenne & 2 jours
\\\cline{2-4}
& En tant que Médecin/Assistant, je veux modifier un patient & Moyenne & 2 jours
\\\cline{2-4}
& En tant que Médecin, je veux supprimer un patient & Basse & 1 jour
\\ \hline
\multirow{8}{8em}{Gérer Rendez-vous}
& En tant qu'Assistant, je veux créer un rendez-vous via POST /api/rendezvous/assistants/\{assistantId\}/patients/\{patientId\}/rdv & Élevée & 4 jours
\\\cline{2-4}
& En tant que Patient, je veux prendre un rendez-vous en ligne avec calendrier et créneaux disponibles & Élevée & 5 jours
\\\cline{2-4}
& En tant qu'Assistant/Patient, je veux consulter les créneaux disponibles via GET /api/rendezvous/medecin/\{medecinId\}/slots-disponibles & Élevée & 3 jours
\\\cline{2-4}
& En tant qu'Assistant, je veux modifier un rendez-vous via PATCH & Moyenne & 2 jours
\\\cline{2-4}
& En tant qu'Assistant, je veux annuler un rendez-vous via DELETE & Moyenne & 2 jours
\\\cline{2-4}
& En tant que Médecin, je veux consulter mon calendrier complet de RDV (créés par assistants et patients) & Élevée & 3 jours
\\\cline{2-4}
& En tant que Patient, je veux consulter mes rendez-vous avec détails (date, médecin, motif, statut) & Moyenne & 2 jours
\\ \hline
\multirow{7}{8em}{Gérer Dossiers Médicaux}
& En tant que Médecin, je veux consulter tous les dossiers via GET /api/dossiers & Moyenne & 2 jours
\\\cline{2-4}
& En tant que Médecin, je veux modifier un dossier existant via PUT /api/dossiers/\{id\} & Moyenne & 3 jours
\\\cline{2-4}
& En tant que Médecin, je veux ajouter des documents (PDF, images) via POST /api/dossiers/\{id\}/files & Élevée & 4 jours
\\\cline{2-4}
& En tant que Médecin, je veux télécharger des documents médicaux associés à un dossier & Moyenne & 2 jours
\\\cline{2-4}
& En tant que Patient, je veux consulter mes dossiers médicaux via GET /api/dossiers/patient/\{patientId\} & Élevée & 3 jours
\\\cline{2-4}
& En tant que Patient, je veux télécharger mes documents médicaux via GET /api/dossiers/\{dossierId\}/files/\{docId\} & Moyenne & 2 jours
\\ \hline
\multirow{7}{8em}{Gérer Factures}
& En tant que Médecin, je veux créer une facture pour n'importe quel patient & Élevée & 3 jours
\\\cline{2-4}
& En tant qu'Assistant, je veux créer une facture UNIQUEMENT pour mes patients liés (vérification automatique) & Élevée & 4 jours
\\\cline{2-4}
& En tant que Médecin/Assistant, je veux enregistrer un paiement via PATCH /api/factures/\{id\}/payer & Élevée & 3 jours
\\\cline{2-4}
& En tant que Médecin, je veux supprimer une facture via DELETE /api/factures/\{id\} (médecin uniquement) & Basse & 1 jour
\\\cline{2-4}
& En tant que Médecin, je veux générer un rapport financier (réservé au médecin) & Moyenne & 4 jours
\\\cline{2-4}
& En tant que Patient, je veux consulter mes factures & Basse & 2 jours
\\ \hline
\multirow{2}{8em}{Chatbot IA}
& En tant que Patient, je veux poser une question au chatbot & Élevée & 5 jours
\\ \hline
\multirow{3}{8em}{Gestion Notifications}
& En tant qu'Utilisateur, je veux consulter mes notifications & Moyenne & 2 jours
\\\cline{2-4}
& En tant qu'Utilisateur, je veux marquer mes notifications comme lues & Basse & 1 jour
\\\cline{2-4}
& En tant qu'Utilisateur, je veux configurer mes préférences de notification & Basse & 2 jours
\\ \hline
\multirow{3}{8em}{Tableau de Bord}
& En tant que Patient, je veux consulter mon tableau de bord avec prochains RDV et accès rapides & Moyenne & 3 jours
\\\cline{2-4}
& En tant que Médecin, je veux consulter mon tableau de bord avec statistiques (patients, RDV, factures) & Élevée & 4 jours
\\\cline{2-4}
& En tant qu'Assistant, je veux consulter mon tableau de bord avec mes statistiques personnelles & Moyenne & 3 jours
\\ \hline
\multirow{6}{8em}{Gestion Profil et Paramètres}
& En tant que Patient, je veux gérer mon profil (nom, prénom, email, téléphone, date naissance, adresse) & Moyenne & 2 jours
\\\cline{2-4}
& En tant que Médecin, je veux gérer mon profil (nom, prénom, email, téléphone, spécialité, description) & Moyenne & 2 jours
\\\cline{2-4}
& En tant qu'Assistant, je veux gérer mon profil (nom, prénom, email, téléphone) & Moyenne & 2 jours
\\\cline{2-4}
& En tant qu'Utilisateur, je veux gérer mes paramètres (préférences notifications, email personnalisé) & Basse & 2 jours
\\\cline{2-4}
& En tant qu'Utilisateur, je veux changer mon mot de passe via /api/account/change-password & Moyenne & 2 jours
\\\cline{2-4}
& En tant qu'Utilisateur, je veux supprimer mon compte via /api/account & Basse & 1 jour
\\ \hline
\end{longtable}


\subsection{Planification des sprints}

La réunion de planification du sprint est une des étapes les plus importantes d'un projet Scrum. L'objectif est de préparer le planning de travail et de choisir les tâches à inclure dans chaque sprint. Après analyse du backlog produit, nous avons décidé de diviser notre projet en trois sprints principaux, précédés d'un sprint 0 d'initialisation : 

\begin{itemize}[leftmargin=2cm, topsep=0pt]
    \begin{spacing}{1.25}
    \item \textbf{Sprint 0 (Initialisation) :} Configuration de l'environnement de développement, setup Spring Boot et Next.js, configuration de la base de données, mise en place de la sécurité JWT et de l'architecture système de base.
    
    \item \textbf{Sprint 1 (Espace Médecin) :} Authentification médecin, gestion complète des assistants (CRUD + activation), gestion du calendrier des rendez-vous, création et modification de dossiers médicaux avec upload de documents, gestion des factures et enregistrement des paiements, génération de rapports financiers.
    
    \item \textbf{Sprint 2 (Espace Assistant) :} Authentification assistant, consultation et modification de patients, création et modification de rendez-vous avec vérification des créneaux disponibles, création de factures avec contrôle d'accès (patients liés uniquement), gestion des notifications et paramètres.
    
    \item \textbf{Sprint 3 (Espace Patient) :} Inscription patient avec vérification email (code à 6 chiffres), authentification patient, consultation des rendez-vous, consultation des dossiers médicaux et téléchargement de documents, chatbot IA pour questions médicales avec intégration OpenAI, gestion des notifications et préférences.
    \end{spacing}
\end{itemize}

\subsection{L'architecture du système}
Pour notre application de gestion de cabinet médical, nous avons choisi une architecture 3-tiers, également appelée architecture à trois niveaux. Cette architecture logicielle bien établie organise l'application en trois niveaux informatiques, logiques et physiques : une couche de présentation (Frontend Next.js), une couche applicative qui traite la logique métier (Backend Spring Boot), et une couche de données qui stocke et gère les informations médicales.

\subsubsection{Couche Présentation (Frontend)}
La couche présentation est développée avec Next.js 14, un framework React moderne. Elle se compose de :
\begin{itemize}[leftmargin=2cm, topsep=0pt]
    \begin{spacing}{1.25}
    \item \textbf{Pages et Composants React :} Organisation modulaire avec App Router, pages dédiées par espace (médecin, assistant, patient), composants réutilisables pour les formulaires et tableaux.
    \item \textbf{Gestion d'État :} Utilisation du Context API pour gérer l'authentification et les informations utilisateur, hooks personnalisés pour l'accès aux données.
    \item \textbf{Communication API :} Appels REST vers le backend Spring Boot, gestion des tokens JWT dans les headers Authorization.
    \item \textbf{Interface Responsive :} Design adaptatif avec Tailwind CSS, compatible desktop, tablette et mobile.
    \end{spacing}
\end{itemize}

\subsubsection{Couche Métier (Backend)}
La couche métier est développée avec Spring Boot 3.5.7 (Java 21) et suit l'architecture MVC enrichie :
\begin{itemize}[leftmargin=2cm, topsep=0pt]
    \begin{spacing}{1.25}
    \item \textbf{Controllers REST :} Exposition d'endpoints API organisés par domaine fonctionnel (UserController, PatientController, MedecinController, AssistantController, RendezVousController, DossierPatientController, FactureController, ChatbotController, NotificationController).
    \item \textbf{Services Métier :} Implémentation de la logique métier complexe (validation des données, calculs, orchestration des opérations).
    \item \textbf{Repositories JPA :} Accès aux données via Spring Data JPA avec requêtes personnalisées.
    \item \textbf{Sécurité :} Spring Security avec JWT, authentification basée sur tokens, autorisation par rôles (ROLE\_PATIENT, ROLE\_MEDECIN, ROLE\_ASSISTANT).
    \item \textbf{Services Transversaux :} EmailService (notifications SMTP), NotificationService (gestion centralisée des notifications), FileStorageService (upload/download de documents), ChatbotService (intégration OpenAI).
    \end{spacing}
\end{itemize}

\subsubsection{Couche Données}
La couche données repose sur une base de données relationnelle (MySQL) et un système de fichiers :
\begin{itemize}[leftmargin=2cm, topsep=0pt]
    \begin{spacing}{1.25}
    \item \textbf{Base de Données :} Tables relationnelles (users, patients, medecins, assistants, rendez\_vous, dossiers\_patients, factures, paiements, notifications, documents), contraintes d'intégrité référentielle, indexation pour optimisation des performances.
    \item \textbf{Stockage Fichiers :} Système de fichiers local pour documents médicaux (ordonnances, analyses), organisation par dossier patient, métadonnées en base de données.
    \end{spacing}
\end{itemize}

\subsubsection{Architecture MVC}
Pendant la réalisation de notre projet, nous avons utilisé l'architecture logicielle MVC (Model-View-Controller), qui décompose l'application en trois composants logiques principaux. Chacun de ces éléments a une fonction spécifique dans le développement de l'application. L'architecture MVC est largement utilisée comme pattern de développement web pour créer des projets extensibles et évolutifs. Nous présentons en détail les trois composants de l'architecture MVC appliqués à notre système :

\begin{itemize}[leftmargin=2cm, topsep=0pt]
    \begin{spacing}{1.25}
    \item \textbf{Modèle (Entités JPA) :} Ce composant correspond à toutes les données relatives à la logique métier. Il contient les entités JPA (User, Patient, Medecin, Assistant, RendezVous, DossierPatient, Facture, Document, Notification) qui mappent les tables de la base de données. Ces entités communiquent avec la base de données via les Repositories pour sauvegarder, consulter ou modifier les données médicales.
    
    \item \textbf{Vue (Frontend Next.js) :} Ce composant représente la couche de présentation de l'application, responsable de fournir l'interface utilisateur (UI). Il se compose d'un ensemble de pages React (dashboard, rendez-vous, dossiers, factures, chatbot) et de composants réutilisables. Il n'intègre aucune logique métier complexe, ces traitements étant gérés par le composant contrôleur backend. La vue communique avec le backend via des appels API REST.
    
    \item \textbf{Contrôleur (Controllers + Services) :} Ce composant contient la logique métier et les algorithmes de traitement. Les Controllers REST exposent les endpoints API et délèguent le traitement aux Services. Par exemple, la vue soumet un formulaire de création de dossier médical au RendezVousController, qui valide les données via le code métier dans DossierPatientService, puis demande au Repository JPA d'effectuer les modifications nécessaires dans la base de données. Le Service gère également les opérations transversales comme l'envoi de notifications et l'upload de documents.
    \end{spacing}
\end{itemize}

\begin{figure}[H]%
    \center%
    \setlength{\fboxsep}{5pt}%
    \setlength{\fboxrule}{0.5pt}%
    \includegraphics[width=16 cm,height=9cm]{images/mvc-cabinet-medical.png}%
    
    \caption{Architecture MVC du Cabinet Médical}%
    
\end{figure}

Le principe de fonctionnement du framework MVC dans notre système de gestion de cabinet médical est le suivant :
\begin{enumerate}[leftmargin=2cm, topsep=0pt]
    \begin{spacing}{1.25}
    \item L'utilisateur (Patient, Médecin ou Assistant) choisit une action à effectuer (ex: créer un rendez-vous) à travers l'interface utilisateur Next.js.
    \item Le Controller backend reçoit la requête HTTP REST, contenant les données JSON du rendez-vous.
    \item Le Controller (@RestController) valide la requête et appelle le Service approprié (RendezVousService) pour traiter la logique métier.
    \item Le Service effectue les vérifications métier (disponibilité créneau, autorisations), interagit avec les Repositories JPA pour accéder aux données (vérifier patient, médecin, créneaux occupés).
    \item Le Modèle (entités JPA) traite la demande en utilisant les données appropriées et effectue les opérations nécessaires (création du rendez-vous en base de données).
    \item Une fois le traitement terminé, le Service peut déclencher des opérations secondaires (envoi notification au patient, email de confirmation).
    \item Le Service renvoie le résultat au Controller sous forme de DTO (Data Transfer Object).
    \item Le Controller encapsule la réponse dans un objet ResponseEntity avec le code HTTP approprié (201 Created, 200 OK, 400 Bad Request, etc.).
    \item La Vue (Frontend Next.js) reçoit la réponse JSON, met à jour l'interface utilisateur et affiche un message de confirmation à l'utilisateur.
    \end{spacing}
\end{enumerate}

\section{CONCLUSION}
Ce chapitre a permis de spécifier de manière détaillée les besoins fonctionnels et non fonctionnels de la plateforme de gestion de cabinet médical intégrant l'intelligence artificielle. L'identification des quatre acteurs principaux (Patient, Médecin, Assistant, Chatbot IA) et de leurs cas d'utilisation respectifs a permis de modéliser le système à travers des diagrammes UML, tandis que la planification avec la méthodologie Scrum a structuré le développement en trois sprints cohérents précédés d'un sprint d'initialisation. L'architecture technique retenue, basée sur des technologies modernes et éprouvées (Spring Boot, Next.js, JWT, OpenAI), offre un socle solide pour la réalisation du projet.

Cette étude préliminaire constitue une base essentielle pour les phases de développement qui seront détaillées dans les chapitres suivants. Elle garantit l'adéquation entre la solution proposée et les besoins exprimés par les différents acteurs du cabinet médical, tout en assurant la qualité, la sécurité (RGPD, secret médical) et la maintenabilité de l'application. La planification détaillée des sprints et la définition claire des livrables permettent d'anticiper les risques et d'assurer le respect des délais du projet académique.
