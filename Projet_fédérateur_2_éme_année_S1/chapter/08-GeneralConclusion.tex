\chapter*{CONCLUSION GÉNÉRALE}
\markboth{\MakeUppercase{GENERAL CONCLUSION}}{}
\addcontentsline{toc}{chapter}{GENERAL CONCLUSION}
\adjustmtc
\thispagestyle{MyStyle}


    Ce projet de développement d’une plateforme de gestion de cabinet médical avec intelligence artificielle a été une expérience enrichissante à la fois sur le plan technique et méthodologique. Nous avons réussi à concevoir et développer un système complet répondant aux besoins réels des professionnels de santé et des patients.

L’adoption de la méthodologie agile SCRUM nous a permis de structurer notre travail en trois sprints progressifs : l’espace médecin (gestion complète du cabinet), l’espace assistant (gestion administrative avec restrictions), et l’espace patient (consultation et chatbot IA). Cette approche incrémentale a facilité la gestion de la complexité du projet et permis de livrer des fonctionnalités opérationnelles à chaque itération.

Sur le plan technique, l’architecture full-stack moderne combinant Next.js 14 pour le frontend et Spring Boot 3.5.7 pour le backend s’est révélée robuste et performante. L’intégration de l’API OpenAI pour le chatbot médical constitue l’innovation majeure du projet, offrant une assistance virtuelle intelligente accessible 24/7 aux patients.

La sécurité a été une préoccupation centrale tout au long du développement. Le système de contrôle d’accès basé sur les rôles (RBAC) garantit la confidentialité des données médicales en limitant strictement les accès selon le rôle de chaque utilisateur (médecin, assistant, patient). Le respect des principes RGPD (hashage des mots de passe, traçabilité des accès, consentement des patients) positionne notre solution comme conforme aux exigences réglementaires du secteur médical.
Les tests réalisés (unitaires, d’intégration, fonctionnels, sécurité) ont permis de valider la fiabilité du système. Les retours positifs lors des démonstrations confirment la pertinence des fonctionnalités développées et l’ergonomie des interfaces.

Ce projet académique nous a permis d’acquérir des compétences techniques approfondies en développement full-stack, en sécurité applicative, et en intégration d’intelligence artificielle. Il nous a également sensibilisés aux enjeux spécifiques du secteur de la santé : confidentialité des données, responsabilité médicale, et importance de la fiabilité des systèmes d’information médicaux.
Les perspectives d’évolution sont nombreuses : prise de rendez-vous en ligne par les patients, téléconsultation, application mobile, amélioration du chatbot avec fine-tuning, et certification HDS pour un déploiement en production réelle.

En conclusion, ce projet démontre qu’il est possible de développer un système de gestion médicale moderne, sécurisé et innovant en combinant les technologies web actuelles avec l’intelligence artificielle. Cette expérience constitue une base solide pour notre future carrière professionnelle dans le développement d’applications d’entreprise critiques.

